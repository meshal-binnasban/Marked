\documentclass[12pt]{article}

\usepackage[margin=2.5cm]{geometry}
\usepackage[T1]{fontenc}
\usepackage[utf8]{inputenc}
\usepackage[hidelinks]{hyperref}
\usepackage{enumitem}

\setlist[itemize]{leftmargin=1.5em}
\setlist[enumerate]{leftmargin=1.5em}

\begin{document}

\title{6-Week Plan: Evaluating, Measuring, and Comparing Algorithms}
\author{}
\date{}
\maketitle

This document outlines a 6-week plan to learn evaluating, measuring, and 
comparing algorithms, going beyond asymptotic complexity 
to include empirical behaviour, structural analysis, and how to
communicate comparisons.

%-----------------------------------------
\section*{Week 1 --- Foundations of Measuring Algorithms}

\subsection*{Goal}
Understand what Big-O gives you and what it does not; get a clear feel for
algorithm growth.

\subsection*{Study}

\begin{itemize}
  \item \textbf{MIT 6.046J: Design and Analysis of Algorithms}
    \begin{itemize}
      \item Course homepage: \\
      \url{https://ocw.mit.edu/courses/6-046j-design-and-analysis-of-algorithms-spring-2015/}
      \item Lecture videos (watch Lecture 1 and Lecture 2): \\
      \url{https://ocw.mit.edu/courses/6-046j-design-and-analysis-of-algorithms-spring-2015/resources/lecture-videos/}
    \end{itemize}

  \item \textbf{Tim Roughgarden --- Algorithms Illuminated (Part 1)}
    \begin{itemize}
      \item Front matter / sample PDF: \\
      \url{https://assets.cambridge.org/97809992/82984/frontmatter/9780999282984_frontmatter.pdf}
      \item Skim the chapter that discusses running time vs.\ practical performance and Big-O basics.
    \end{itemize}
\end{itemize}

\subsection*{Tiny writing task}

Write a short note for yourself (5--10 sentences):
\begin{quote}
  What aspects of an algorithm matter besides asymptotic complexity?
\end{quote}

%-----------------------------------------
\section*{Week 2 --- Practical Algorithm Engineering \& Real Engines}

\subsection*{Goal}
See how real systems people talk about algorithm behaviour and performance.

\subsection*{Study}

\begin{itemize}
  \item \textbf{Russ Cox --- ``Regular Expression Matching Can Be Simple and Fast''}
    \begin{itemize}
      \item Article: \\
      \url{https://swtch.com/~rsc/regexp/regexp1.html}
    \end{itemize}

  \item \textbf{Rust regex engine documentation}
    \begin{itemize}
      \item Crate documentation: \\
      \url{https://docs.rs/regex/latest/regex/}
      \item Crate overview (guarantees, discussion): \\
      \url{https://crates.io/crates/regex}
    \end{itemize}

  \item \textbf{Algorithm Engineering / Empirical Slides (Meyer)}
    \begin{itemize}
      \item Slides PDF: \\
      \url{https://people.mpi-inf.mpg.de/~mehlhorn/AlgorithmEngineering/ExternalMemorySlides.pdf}
    \end{itemize}
\end{itemize}

\subsection*{Tiny writing task}

Write six bullet points answering:
\begin{quote}
  What practical problems do real regex engines face, and how might a marked
  approach help?
\end{quote}

%-----------------------------------------
\section*{Week 3 --- Structure-Driven Behaviour (Your Regex World)}

\subsection*{Goal}
Understand structurally why algorithms blow up (derivatives vs.\ marks).

\subsection*{Study}

\begin{itemize}
  \item \textbf{Your own report (ReportV3)}
    \begin{itemize}
      \item Re-read the sections on the derivative-based matcher and the marked matcher,
            focusing on where you discuss size explosion and mark growth.
    \end{itemize}

  \item \textbf{Sulzmann \& Lu --- POSIX Regular Expression Parsing with Derivatives}
    \begin{itemize}
      \item Springer chapter: \\
      \url{https://link.springer.com/chapter/10.1007/978-3-319-07151-0_13}
      \item Preprint PDF: \\
      \url{https://arxiv.org/pdf/1604.06644}
    \end{itemize}
\end{itemize}

\subsection*{Tiny writing task}

Create a small table (for yourself) with the columns:
\begin{center}
  Construct \quad | \quad Growth in derivatives \quad | \quad Growth in marks
\end{center}
Fill it for \texttt{SEQ}, \texttt{STAR}, \texttt{ALT}, and \texttt{NTIMES}, noting what
structurally causes growth in each representation.

%-----------------------------------------
\section*{Week 4 --- Statistical and Experimental Thinking}

\subsection*{Goal}
Move from ``I ran it once'' to systematic empirical evaluation.

\subsection*{Study}

\begin{itemize}
  \item \textbf{Intro to performance analysis (lecture notes)}
    \begin{itemize}
      \item Example notes inspired by Raj Jain: \\
      \url{https://www.cs.rice.edu/~johnmc/comp528/lecture-notes/Lecture1.pdf}
    \end{itemize}

  \item \textbf{Raj Jain --- The Art of Computer Systems Performance Analysis}
    \begin{itemize}
      \item Book on archive.org: \\
      \url{https://archive.org/details/artofcomputersys0000jain}
      \item Skim the early chapters on measurement and experimental design.
    \end{itemize}

  \item \textbf{(Optional) Methodology of Algorithm Engineering}
    \begin{itemize}
      \item arXiv PDF: \\
      \url{https://arxiv.org/pdf/2310.18979}
    \end{itemize}
\end{itemize}

\subsection*{Tiny writing task}

Design a small benchmark for your own matchers and write down:
\begin{itemize}
  \item Which regular expressions you will test.
  \item Which input lengths you will use.
  \item Which metrics you will record (e.g.\ time, memory, mark count, derivative size).
\end{itemize}

%-----------------------------------------
\section*{Week 5 --- Learning from Evaluation Sections in Papers}

\subsection*{Goal}
Copy the style of evaluation sections from real papers in your area.

\subsection*{Study}

\begin{itemize}
  \item \textbf{Sulzmann \& Lu --- POSIX Regular Expression Parsing with Derivatives}
    \begin{itemize}
      \item Focus on the evaluation / performance discussion: \\
      \url{https://link.springer.com/chapter/10.1007/978-3-319-07151-0_13}
    \end{itemize}

  \item \textbf{Tan \& Urban --- POSIX Lexing with Bitcoded Derivatives}
    \begin{itemize}
      \item DROPS entry: \\
      \url{https://drops.dagstuhl.de/entities/document/10.4230/LIPIcs.ITP.2023.27}
      \item Direct PDF: \\
      \url{https://drops.dagstuhl.de/storage/00lipics/lipics-vol268-itp2023/LIPIcs.ITP.2023.27/LIPIcs.ITP.2023.27.pdf}
    \end{itemize}

  \item \textbf{Urban --- POSIX Lexing with Derivatives of Regular Expressions}
    \begin{itemize}
      \item PDF: \\
      \url{https://urbanchr.github.io/Publications/posix.pdf}
    \end{itemize}

  \item \textbf{Might, Darais, Spiewak --- Parsing with Derivatives}
    \begin{itemize}
      \item PDF: \\
      \url{https://matt.might.net/papers/might2011derivatives.pdf}
    \end{itemize}
\end{itemize}

\subsection*{Tiny writing task}

For each of the above papers, write three bullet points:
\begin{quote}
  What do they measure, and how do they explain why the algorithm behaves that way?
\end{quote}

%-----------------------------------------
\section*{Week 6 --- Synthesising Theory, Empirics, and Explanation}

\subsection*{Goal}
Be able to answer viva-style questions such as:
\begin{quote}
  Why is your algorithm better here? \\
  What causes slow behaviour in your approach? \\
  What evidence supports your claims?
\end{quote}

\subsection*{Study}

No new resources this week; instead:
\begin{itemize}
  \item Review your notes from Weeks 1--5.
  \item Review your benchmarks and plots.
  \item Revisit key parts of the papers above as needed.
\end{itemize}

\subsection*{Final exercises}

\begin{itemize}
  \item \textbf{Four-component comparison.} For both the derivative-based and
        marked matchers, write:
        \begin{itemize}
          \item Asymptotic complexity (what is known or conjectured).
          \item Structural behaviour (which regex structures cause trouble).
          \item Empirical behaviour (what your tests showed).
          \item Semantic behaviour (POSIX values, determinism, formalisation).
        \end{itemize}

  \item \textbf{Viva rehearsal.} Practise answering out loud:
        \begin{itemize}
          \item How did you compare the two algorithms?
          \item What do your experiments tell you, and what do they \emph{not} tell you?
          \item In which cases could your marked approach also be slow, and why?
        \end{itemize}
\end{itemize}

\end{document}
