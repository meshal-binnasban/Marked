\documentclass[12pt]{article}

% Basic math packages
\usepackage{amsmath}
\usepackage{amssymb}
\usepackage{amsthm}

\usepackage[T1]{fontenc}
\usepackage[utf8]{inputenc}

% Theorem environments
\newtheorem{lemma}{Lemma}

\begin{document}

%-------------------- Proof of correctness of shifts for r^n --------------------
\section*{Proof of :$r^n$}
We want to prove, for every $n \ge 0$ and every set of strings $ms$:
\[
P(n, ms) \;:\;\; \mathit{shifts}(ms, r^n) \;=\; \mathit{Strips}(L(r^n))(ms).
\]
We consider the following:
\[
L(r^0) = \emptyset,\qquad L(r^1) = L(r),\qquad L(r^{n+1}) = L(r) @ L(r^n).
\]

\section{Case $n = 0$.}

By unfolding the definition:
\[
\mathit{shifts}(ms, r^0) = \emptyset.
\]
Since $L(r^0) = \emptyset$,
\[
\mathit{Strips}(L(r^0))(ms) = \mathit{Strips}(\emptyset)(ms) = \emptyset.
\]
Thus
\[
\mathit{shifts}(ms, r^0) = \mathit{Strips}(L(r^0))(ms),
\]
Therefore $P(0,ms)$ holds.

\section{Case $n = 1$.}

By unfolding the definition:
\[
\mathit{shifts}(ms, r^1) = \mathit{shifts}(ms, r).
\]
By IH:
\[
\mathit{shifts}(ms, r) = \mathit{Strips}(L(r))(ms) = \mathit{Strips}(L(r^1))(ms).
\]
Therefore $P(1,ms)$ holds.

\section{Case $n > 1$.}

we split the proof by two cases of $\mathit{shifts}(ms, r)$.

\subsection{$\mathit{shifts}(ms, r) = \emptyset$.}
\label{caseA}
By unfolding the definition:
\[
\mathit{shifts}(ms, r^n) = \emptyset.
\]
Since
\[
\mathit{shifts}(ms, r) = \emptyset
\quad\Rightarrow\quad
\mathit{Strips}(L(r))(ms)=\emptyset.
\]
If $\mathit{Strips}(L(r))(ms)=\emptyset$, then
\[
\mathit{Strips}(L(r^n))(ms)=\emptyset.
\]
So
\[
\mathit{shifts}(ms, r^n) = \emptyset = \mathit{Strips}(L(r^n))(ms),
\]
Therefore $P(n,ms)$ holds in Subcase~\ref{caseA}.

\subsection{$\mathit{shifts}(ms, r) \neq \emptyset$.}
\label{caseB}
Before unfolding further, introduce repeated stripping:
\[
F_r(X) = \mathit{Strips}(L(r))(X)
\]
for any set $X$, and define:
\[
F_r^1(X)=F_r(X),\qquad F_r^{k+1}(X)=F_r(F_r^k(X)).
\]
So:
\[
\mathit{shifts}(ms, r) = \mathit{Strips}(L(r))(ms) = F_r(ms).
\]
We now have two further cases according to $\mathit{nullable}(r)$.

\subsubsection{$\neg \mathit{nullable}(r)$.}
\label{subcaseb1}
by unfolding shifts defintion:
\[
\mathit{shifts}(ms, r^n)
  = \mathit{shifts}\big(\mathit{shifts}(ms, r),\ r^{n-1}\big).
\]
By IH: 
\[
\mathit{shifts}\big(\mathit{shifts}(ms,r), r^{n-1}\big)
  = \mathit{Strips}(L(r^{n-1}))\big(\mathit{Strips}(L(r))(ms)\big).
\]
Using $\mathit{Strips}(L(r))(ms) = F_r(ms)$, we get
\[
\mathit{shifts}(ms, r^n)
  = \mathit{Strips}(L(r^{n-1}))\big(F_r(ms)\big).
\]
By Strips properties:
with not nullable A, and B
 \[Strips (A\,@\, B) C= Strips\, B\, (Strips\, A\, C)\]
\[A=L(r) \texttt{ and } B=L(r^{n-1})  \]

\begin{tabular}{rl}
    \\
    $\mathit{Strips}(L(r^n))(ms)=$&$ \mathit{Strips}(L(r) @ L(r^{n-1}))(ms)$\\
    $=$& $\mathit{Strips}\big(L(r^{n-1}), \mathit{Strips}(L(r), ms)\big)$\\
    $=$&$\mathit{Strips}(L(r^{n-1}))(F_r(ms)).$\\
\end{tabular}

So
\[
\mathit{shifts}(ms, r^n)
 = \mathit{Strips}(L(r^n))(ms),
\]
Therefore $P(n,ms)$ holds in Subcase~\ref{subcaseb1}.


\subsection{$\mathit{nullable}(r)$.}
\label{subcaseb2}
By unfolding shifts defintion:
\[
\mathit{shifts}(ms, r^n)
 = \mathit{shifts}(ms,r)\ \cup\ \mathit{shifts}\big(\mathit{shifts}(ms,r),\ r^{n-1}\big),
\]
By IH on $\mathit{shifts}(ms,r) = \mathit{Strips}(L(r))(ms)= F_r(ms)$, we get
\[
\mathit{shifts}(ms, r^n)
 = F_r(ms) \ \cup\ \mathit{shifts}\big(F_r(ms),\ r^{n-1}\big).
\]
By IH applied to  $n-1$,
\[
\mathit{shifts}\big(F_r(ms), r^{n-1}\big)
  = \mathit{Strips}(L(r^{n-1}))\big(F_r(ms)\big).
\]
then the unfolded definition becomes:
\[
\mathit{shifts}(ms, r^n)
 = F_r(ms) \cup \mathit{Strips}(L(r^{n-1}))\big(F_r(ms)\big).
\]

We now use:

\begin{lemma}[nullable $r^m$]
\label{lem:nullable-Fr}
If $\mathit{nullable}(r)$, then for every $m \ge 1$ and every set $X$,
\[
\mathit{Strips}(L(r^m))(X)
 = \bigcup_{k=1}^m F_r^k(X).
\]
by induction on $m$, using all nullable concatenation property of Strips. (at end of the file)

\end{lemma}

\medskip

Apply Lemma~\ref{lem:nullable-Fr} with $m = n-1$ and $X = F_r(ms)$:
\[
\mathit{Strips}(L(r^{n-1}))(F_r(ms))
 = \bigcup_{k=1}^{n-1} F_r^k(F_r(ms))
 = \bigcup_{k=2}^{n} F_r^k(ms).
\]
So
\[
\mathit{shifts}(ms, r^n)
 = F_r(ms) \cup \bigcup_{k=2}^{n} F_r^k(ms)
 = \bigcup_{k=1}^n F_r^k(ms).
\]
Applying Lemma~\ref{lem:nullable-Fr} again with $m = n$ and $X = ms$,
\[
\mathit{Strips}(L(r^n))(ms)
 = \bigcup_{k=1}^n F_r^k(ms).
\]
So
\[
\mathit{shifts}(ms, r^n) = \mathit{Strips}(L(r^n))(ms),
\]
Therefore $P(n,ms)$ holds in Subcase~\ref{subcaseb2}.

\medskip

Since all cases $n=0$, $n=1$, and $n>1$ (with subcases) have been included,
so:
\[
\mathit{shifts}(ms, r^n) = \mathit{Strips}(L(r^n))(ms).
\]
\newpage




\section{Detailed proof of Lemma~\ref{lem:nullable-Fr}}
\label{sec:detailed-lemma-proof}

\begin{proof}[Proof of Lemma~\ref{lem:nullable-Fr}]
By induction on $m$.

\paragraph{Base case $m = 1$.}

Since $L(r^1) = L(r)$, we have
\[
  \mathit{Strips}(L(r^1))(X)
    = \mathit{Strips}(L(r))(X)
    = F_r(X)
    = \bigcup_{k=1}^1 F_r^k(X).
\]
So the statement holds for $m = 1$.

\paragraph{Induction step.}

Assume for some $m \ge 1$ and all sets $X$,
\[
  \mathit{Strips}(L(r^m))(X)
    = \bigcup_{k=1}^m F_r^k(X).
\]
We must show, for all $X$, that:
\[
  \mathit{Strips}(L(r^{m+1}))(X)
    = \bigcup_{k=1}^{m+1} F_r^k(X).
\]
recall:
\[
  L(r^{m+1}) = L(r) @ L(r^m).
\]

Since lemma is for $\mathit{nullable}(r)$, so $[] \in L(r)$.
and for $r^m$ we also have $[] \in L(r^m)$.
So we may apply the all nullables concatenation properties of Strips with
$A = L(r)$, $B = L(r^m)$, $C = X$:

\begin{align}
  \mathit{Strips}(L(r^{m+1}))(X)
    &= \mathit{Strips}(L(r) @ L(r^m))(X) \notag \\
    &= \mathit{Strips}(L(r^m), \mathit{Strips}(L(r), X)) \notag \\
    &\quad \cup\; \mathit{Strips}(L(r), X) \notag \\
    &\quad \cup\; \mathit{Strips}(L(r^m), X).  \label{eq:strips-concat}
\end{align}



By IH of the lemma:
\[
  \mathit{Strips}(L(r^m), X)
    = \bigcup_{k=1}^m F_r^k(X).
\]

We use  $F_r(X)$ :
\[
  \mathit{Strips}(L(r), X) = F_r(X).
\]

Also, by the IH:
\[
\begin{aligned}
  \mathit{Strips}(L(r^m), \mathit{Strips}(L(r), X))
    &= \mathit{Strips}(L(r^m), F_r(X)) \\
    &= \bigcup_{k=1}^m F_r^k(F_r(X)) \\
    &= \bigcup_{k=1}^m F_r^{k+1}(X) \\
    &= \bigcup_{j=2}^{m+1} F_r^j(X).
\end{aligned}
\]

then (\ref{eq:strips-concat}) becomes:
\[
\begin{aligned}
  \mathit{Strips}(L(r^{m+1}))(X)
    &= \left( \bigcup_{j=2}^{m+1} F_r^j(X) \right)
       \cup F_r(X)
       \cup \left( \bigcup_{k=1}^{m} F_r^k(X) \right).
\end{aligned}
\]

Since $F_r(X) = F_r^1(X)$, and already contained in $\bigcup_{k=1}^{m} F_r^k(X)$.

\medskip

The union of $\, \bigcup_{k=1}^{m} F_r^k(X)$ and $\bigcup_{j=2}^{m+1} F_r^j(X)$
contains the sets $F_r^k(X)$ for $k = 1,2,\dots,m+1$.
Therefore
\[
  \mathit{Strips}(L(r^{m+1}))(X)
    = \bigcup_{k=1}^{m+1} F_r^k(X).
\]

This completes the induction step and the proof of the lemma.
\end{proof}


\end{document}
