\documentclass[12pt]{article}
\usepackage{amsmath}
\usepackage{amssymb}
\usepackage{amsthm}
\usepackage[section]{placeins}
\usepackage{tikz}
\usepackage{pgf}
\usepackage{pgfplots}
\pgfplotsset{compat=1.15}

\DeclareMathSizes{12}{10}{8}{6}

\title{Progress Report - 9 Months}
\author{Meshal Binnasban}
\date{\today}

\begin{document}
%define =
\newcommand{\dn}{\ensuremath{\stackrel{\mbox{\scriptsize def}}{=}}}
%regex
\newcommand{\ZERO}{\small\textbf{0}}   
\newcommand{\ONE}{\small\textbf{1}}    

%function definitions
\newcommand{\der}{\textit{der}}
\newcommand{\shift}{\textit{shift}}
\newcommand{\shifts}{\textit{shifts}}
\newcommand{\fuse}{\textit{fuse}}
\newcommand{\mkeps}{\textit{mkeps}}
\newcommand{\intern}{\textit{intern}}
\newcommand{\mkfin}{\textit{mkfin}}
\newcommand{\fin}{\textit{fin}}
\newcommand{\nullable}{\textit{nullable}}
\newcommand{\matcher}{\textit{matcher}}
\newcommand{\reshuffle}{\textit{reshuffle}}
\newcommand{\follow}{\textit{follow}}
\newcommand{\readF}{\textit{read}}
\newcommand{\move}{\textit{move}}
%value definitions
\newcommand{\Seq}{\textit{Seq}}
\newcommand{\Left}{\textit{Left}}
\newcommand{\Right}{\textit{Right}}
\newcommand{\Star}{\textit{Star}}
\newcommand{\Empty}{\textit{Empty}}

%symbols
\newcommand{\emptylist}{[\,]}
\newcommand{\At}{\text{$\,@\,$}}
\newcommand{\Marked}[1]{\bullet\,#1}
\newcommand{\Bits}{\textit{Bits}}
\newcommand{\POINT}{\textit{POINT}}
\newcommand{\bigO}[1]{\ensuremath{\mathop{}\mathopen{}\mathcal{O}\mathopen{}\left(#1\right)}}
\newcommand{\sizeExplosion}{"size explosion"}


\maketitle

\begin{abstract}
This is the nine-month progress report for my PhD at King’s College London.
It summarises the motivation, challenges, and progress made in developing a
regular expression matcher based on marked regular expressions,
with the goal of POSIX value extraction.
The report first reviews regular expression matching using derivatives, then
reviews marked regular expression matchers, which presently provide matching
but not POSIX value extraction—our primary focus.
Our work developed several versions of the marked algorithm in Scala.
We have also begun formally proving the correctness of one of our versions.
\end{abstract}

\newpage

\tableofcontents

\newpage

\section{Introduction}
The notion of derivatives in regular expressions was introduced by
Brzozowski in 1964~\cite{Brzozowski1964}, but it was rediscovered in
the 2000s and gained renewed attention—for example in
the work of Owens~\cite{Owens2009} and Might~\cite{Might2011}.
This renewed interest is largely due to the method’s conceptual simplicity 
and the ease with which it can be implemented in functional programming 
languages.
%The notion of derivatives in regular expressions is well established 
%but has gained renewed attention in the last decade, for example in 
%the work of Owens~\cite{Owens2009} and Might~\cite{Might2011}. Their 
%simplicity and compatibility with functional programming have 
%encouraged further studies, for example the work of Sulzmann and 
%Lu~\cite{Sulzmann2014}.  
However, derivatives suffer from \sizeExplosion, since taking 
derivatives may increase the size of subexpressions. 
This means the algorithm has to traverse larger and larger 
regular expressions, which results in slower performance.

In contrast, the approach based on marks leaves regular expressions unchanged 
during matching but moves annotations through them. 
At present, only matching algorithms using this approach exist, 
while our work aims to extend them to provide POSIX value extraction—the value 
representing the longest leftmost match.

This report first reviews Brzozowski's derivatives and the bit-coded variant of
Sulzmann and Lu~\cite{Sulzmann2014}, followed by background on the regular
expression matchers based on marks described by Fischer et~al.~\cite{Fischer2010}
and Asperti et~al.~\cite{Asperti2010}.
We then present our own work, including several versions of matchers based on
marks developed during the first nine months.

\section{Background}
Regular expressions are a way to describe languages of strings over an alphabet.
They provide a declarative way to specify such sets of strings. Each regular
expression $r$ is associated with a language $L(r)$, which is the set of strings
that match $r$. The regular expressions we consider are the standard regular 
expression constructors, including bounded repetitions, namely:
\[
r ::= \ZERO \;\mid\; \ONE \;\mid\; c \;\mid\; r_1 + r_2 \;\mid\; r_1 \cdot r_2 \;\mid\; 
r^{*} \;\mid\; r^{n}
\]

We will use \ZERO\ for the regular expression that cannot match any string,
\ONE\ for the regular expression that can match the empty string, and $c$
for the regular expression that can match a character from the alphabet.
The operator $+$ is for alternation of $r_1$ and $r_2$, while the
dot~$\cdot$ is for concatenation or sequencing.
The Kleene star $r^{*}$ is for arbitrary repetition of $r$, and the power
$r^n$ is for exactly $n$ repetitions of $r$.
The meaning of these regular expressions is given by the language function
$L$ (Figure~\ref{lFunction}).

\begin{figure}[ht]
\begin{center}
%Language function $L$%
  \renewcommand{\arraystretch}{1.5}
  \begin{tabular}{lcl}
    $L(\ZERO )$            & \dn & $\ZERO$ \\
    $L(\ONE )$             & \dn & $\{\emptylist\}$ \\
    $L(c)$                  & \dn & $\{[c]\}$ \\
    $L(r_1 + r_2)$          & \dn & $L(r_1) \cup L(r_2)$ \\
    $L(r_1 \cdot r_2)$      & \dn & $L(r_1) \cdot L(r_2)$ \\
    $L(r^{*})$             & \dn & $(L(r))^{*}$ \\
    $L(r^n)$                & \dn & $(L(r))^n$
  \end{tabular}
  \renewcommand{\arraystretch}{1.0}
     \rule{\linewidth}{0.4pt}
     \caption{the function $L$, which gives meaning to a regular expression,
        i.e.\ the set of strings that belong to its language.}\label{lFunction}
  \end{center}
\end{figure}
For example, if $r = a + b$, then
\[
L(r) = L(a) \cup L(b) = \{a\} \cup \{b\} = \{a, b\}.
\]
\noindent Similarly, $L(a \cdot b) = \{ab\}$, and $L(a^{*})$ is the set of all strings 
consisting of zero or more $a$'s.
\FloatBarrier
\subsection{Derivatives}
Brzozowski’s derivatives offer an elegant way for regular expression matching 
(Figure \ref{BrzozowskiDerivative}). What is beautiful about derivatives is that 
they can be easily implemented in a functional programming language and are 
easy to reason about in theorem provers. By successively taking the derivative 
of a regular expression with respect to each input character, one obtains a 
sequence of derivatives. If the final derivative can match the empty string, 
then the original input is accepted. For example, to decide whether a string 
$a_1 a_2 \dots a_n$ is matched by a regular expression $r$, we successively 
compute derivatives:
\[
\begin{array}{rcl}
r_0 = r,& r_1 = \der_{a_1}(r_0),& r_2=\der_{a_2}(r_1) \dots, r_n = \der_{a_n}(r_{n-1}).
\end{array}
\]
The string matches $r$ if and only if the final expression $r_n$ accepts the
empty string, which can be easily checked by a separate function.
Here $\der_a(r)$ stands for the derivative of $r$ with respect to the
character $a$.
%Brzozowski’s Derivative
\begin{figure}[ht]

  \begin{center}
  \renewcommand{\arraystretch}{1.5}
  \begin{tabular}{lcl}
    $\der_a(\ZERO)$            & \dn & $\ZERO$ \\
    $\der_a(\ONE)$             & \dn & $\ZERO$ \\
    $\der_a(c)$                  & \dn &
      $\begin{cases}
        \ONE  & \text{if } a = c \\
        \ZERO & \text{if } a \neq c
      \end{cases}$ \\
    $\der_a(r_1 + r_2)$          & \dn & $\der_a(r_1) + \der_a(r_2)$ \\
    $\der_a(r_1 \cdot r_2)$      & \dn &
      $\der_a(r_1) \cdot r_2 \;+\;
        \begin{cases}
          \der_a(r_2) & \text{if } \varepsilon \in L(r_1) \\
          \ZERO       & \text{otherwise}
        \end{cases}$ \\
    $\der_a(r^{*})$             & \dn & $\der_a(r) \cdot r^{*}$
  \end{tabular}
  \renewcommand{\arraystretch}{1.0}

    \mbox{}
    \rule{\linewidth}{0.4pt}
  \caption{Brzozowski’s Derivative.}\label{BrzozowskiDerivative}
  \end{center}
\end{figure}

To illustrate how derivatives can be used to match a regular expression 
against a string, consider the regular expression $( ab + ba )$. 
The derivative can check the matching of the string \texttt{ba} by taking 
the derivative of the regular expression with respect to $b$, then with respect 
to $a$ (see Figure~\ref{derivative}). Since the final derivative expression 
can match the empty string with no input characters remaining, this confirms that 
the string \texttt{ba} belongs to the language of the regular expression.
\begin{figure}[ht]
  
  \begin{center}
\[
\begin{array}{rcl}
\der_b\, r           & =           &  \der_b\, (ab + ba) \\
                     & =           & \der_b\, (ab) + \der_b\, (ba) \\
                     & =           & (\der_b\, a) \cdot b + (\der_b\, b) \cdot a  \\
                     & =           & \ZERO \cdot b + \ONE \cdot a \\\\

\der_a\, (\der_b\, r) & =           & \der_a\, (\ZERO \cdot b + \ONE \cdot a)\\
                      & =           & \der_a\, (\ZERO \cdot b) + \der_a\, (\ONE \cdot a)\\
                      & =           & \der_a\, (\ZERO) \cdot b + (\der_a\, (\ONE) \cdot a + \der_a\, a)\\
                      & =           & \ZERO \cdot b + (\ZERO \cdot a + \ONE )
\end{array}
\]
    \mbox{}
    \rule{\linewidth}{0.4pt}
\caption{Derivatives of $(ab+ba)$ with respect to string 
\texttt{ba}.}\label{derivative}
  \end{center}
\end{figure}

Note, however, that some subexpressions that arise during the computation
of derivatives may be redundant. For example, $\ZERO  \cdot b$ cannot match
any string, so we could simplify it to \ZERO, which can match the same
strings—namely none (see Figure~\ref{derivative}).
Other simplification rules may also be applied. These include associativity
$(r+s)+t \equiv r+(s+t)$, commutativity $r+s \equiv s+r$, and idempotence
$r+r \equiv r$. Rules for \ZERO\ and \ONE\ can also be applied; e.g.\
$\ZERO \cdot r \equiv r \cdot \ZERO \equiv \ZERO$, $r+\ZERO \equiv r$,
and $r \cdot \ONE \equiv \ONE \cdot r \equiv r$.

These simplifications preserve the language of the regular expression
while reducing its overall size.
Such simplifications are sometimes necessary in the derivative method
to keep the size manageable, but even with a best effort,
the size can still grow significantly.
\FloatBarrier
\subsubsection{Size Explosion}  
Although the construction of derivatives is elegant, it may
duplicate large parts of the expression and increase the size of the regular
expression in intermediate steps (see Figure~\ref{BrzozowskiDerivative},
the sequence case can create additional subexpressions, as can the star case).
Sulzmann and Lu~\cite{Sulzmann2014} also note this problem, describing it as the
well-known issue that the size and number of derivatives may explode.
The size of derivatives can become very large even for simple expressions.
This happens because each derivative is usually more complicated than the
original expression, and even when standard simplifications are applied,
they may not remove all duplications.
Consider, for example, the regular expression $(a+aa)^{*}$ and a derivative
with respect to a single $a$.
\[
\begin{array}{rcl}
\der_a \, (a + aa)^{*} & = & (\der_a \, (a+aa)) \cdot (a+aa)^{*} \\
                       & = & (\der_a \, a + \der_a \, (aa)) \cdot (a+aa)^{*} \\
                       & = & (\ONE + \der_a \, (a \cdot a)) \cdot (a+aa)^{*} \\
                       & = & (\ONE + (\ONE \cdot a)) \cdot (a+aa)^{*} \\
                       & = & (\ONE + a) \cdot (a+aa)^{*}
\end{array}
\]
Each derivative step may introduce new structure, unfolding all
possible ways in which $r^{*}$ can match the input.
%Even with basic simplifications, many of the resulting subexpressions 
%still describe thesame language but in different syntactic forms, and so 
%the size continues togrow.

It is already known that the simplification reduces the number of generated
expressions and helps to provide a finite bound on the number of intermediate
expressions~\cite{Sulzmann2014,TanAndUrban2023}, but it does not completely
solve the underlying problem. Even with aggressive simplifications---as in
Sulzmann and Lu’s bit-coded approach, and in the variant described by Tan and
Urban---the resulting derivatives can still be large even if finitely bounded.
For example, expressions of the form
$((a)^{*} + (aa)^{*} + (aaa)^{*} + \dots )^{*}$
cannot be simplified completely, since the duplicates occur at different
levels of the derivatives and across sequences and alternatives.
Thus, even under aggressive simplification, the size of the derivatives
remains large and continues to grow across derivative steps.
If we consider the regular expression
\[
((a)^{*} + (aa)^{*} + (aaa)^{*} + (aaaa)^{*} + (aaaaa)^{*})^{*}
\]
with a small input string of the form $\underbrace{a \dots a}_n$, 
each derivative step introduces subexpressions that are copies of the original,
as shown in Figure~\ref{derivativeStarsExample}.
\begin{figure}[ht]
  
  \begin{center}
    \[
    \begin{array}{rcl}
      \der_a\, r          & = &  \der_a\,( ((a)^{*} + (aa)^{*} + (aaa)^{*} + (aaaa)^{*} + (aaaaa)^{*} )^{*}) \\
                          & = & \der_a\, ((a)^{*} + (aa)^{*} + (aaa)^{*} + (aaaa)^{*} + (aaaaa)^{*} ) \cdot \, r^{*} \\  
                          & = & ((\der_a\, (a)^{*}) + (\der_a\, (aa)^{*})+ (\der_a\,(aaa)^{*}) + \dots ) \cdot  r^{*} \\
                          & = & ((\der_a\, a \cdot (a)^{*}) + (\der_a\,aa \cdot (aa)^{*}) + (\der_a\,aaa\cdot (aaa)^{*})+ \dots ) \cdot  r^{*} \\
                          & = & ((\ONE \cdot (a)^{*}) + ((\ONE \cdot a) \cdot (aa)^{*})+ ( (\ONE \cdot aa) \cdot (aaa)^{*}) + \dots ) \cdot  r^{*}\\
                          & = & ((a)^{*} + (a \cdot (aa)^{*})+ ( aa \cdot (aaa)^{*}) + \dots ) \cdot  r^{*} \\\\

      \der_a\, (\der_a\, r) & = & \der_a\, ((a)^{*} + (a \cdot (aa)^{*})+( aa \cdot (aaa)^{*}) + \dots ) \cdot r^{*} + \der_a\,r^{*} \\
                            & = & (\ONE \cdot (a)^{*} + (\ONE \cdot (aa)^{*})+ ( (\ONE \cdot a) \cdot (aaa)^{*}) +  \dots ) \cdot r^{*} \,+ \\
                            &   & \quad ((a)^{*} + (a \cdot (aa)^{*})+ ( aa \cdot (aaa)^{*}) + \dots ) \cdot  r^{*} \\ 
                            & = & ((a)^{*} + (aa)^{*} + ( a \cdot (aaa)^{*}) + \dots ) \cdot r^{*} \,+ \\
                            &   & \quad ((a)^{*} + (a \cdot (aa)^{*})+ ( aa \cdot (aaa)^{*}) + \dots ) \cdot  r^{*} \\\\ 
                            
                        
     \der_a\,( \der_a\, (\der_a\, r)) & =           & \der_a\, (((a)^{*} + (aa)^{*}+ ( a \cdot (aaa)^{*}) +  \dots ) \cdot r^{*} \,)+ \\
                            &             & \quad  \der_a\,(((a)^{*} + (a \cdot (aa)^{*})+ ( aa \cdot (aaa)^{*}) +  \dots ) \cdot  r^{*} )  \\
                            & =           & (\der_a\, (((a)^{*} + ((aa)^{*})+ ( a \cdot (aaa)^{*}) + \dots ) \cdot r^{*}) \, + \der_a\, r^{*} )+\\
                            &             & \quad ( \der_a\,(((a)^{*} + (a \cdot (aa)^{*})+( aa \cdot (aaa)^{*}) +\dots ) \cdot  r^{*}) + \der_a\, r^{*})                     
            \end{array}
    \]
    \mbox{}
    \rule{\linewidth}{0.4pt}
\caption{Derivatives of $((a)^{*}+(aa)^{*}+(aaa)^{*}+(aaaa)^{*}+(aaaaa)^{*})^{*}$
with respect to the string \texttt{aaa}.
Each step introduces additional copies of the original regular expression
at different levels of the derivative, where simplifications cannot be applied.
For example, after the second step, $(a)^{*}$ appears in both alternatives,
but, since they occur under different concatenations with $r$, the two
occurrences cannot be merged.}
\label{derivativeStarsExample}
  \end{center}
\end{figure}

Figure~\ref{dervsmarGraph} illustrates the \sizeExplosion\ of derivatives by
showing the runtime difference between derivatives with aggressive simplifications
and Fischer’s marked algorithm. We conducted a naive test using a regular
expression of the kind mentioned above to demonstrate the impact of the
\sizeExplosion\ of derivatives, even under aggressive simplifications.
Although the test is naive, it clearly shows a considerable difference in
behaviour. The test was designed for inputs of up to $100{,}000$ characters;
however, the derivatives implementation ran out of memory after $10{,}000$,
with the last successful case taking almost $22$ seconds, while the marked
algorithm finished executing all inputs, with each case under $0.0035$ seconds.
\begin{figure}[ht]
  
\mbox{}\bigskip
\begin{center}
\begin{tikzpicture}
\begin{axis}[
    xlabel={$n$},
    x label style={at={(1.05,0.0)}},
    ylabel={time in secs},
    enlargelimits=true,
    xtick={0,2500,...,10000},
    %xtick distance=2000,
    xmax=10000,
    ymax=30,
   % ytick={0,0.0005,...,0.0035},
    ytick={0,2,...,30},
    scaled ticks=true,
    axis lines=left,
    width=8cm,
    height=8cm, 
    legend entries={Marks,Derivatives},  
    legend pos=north west,
    legend cell align=left]
\addplot[blue,mark=*, mark options={fill=white}] table {play.data};
\addplot[cyan,mark=*, mark options={fill=white}] table {DerivSimp.data};

\end{axis}
\end{tikzpicture} 
    \mbox{}
    \rule{\linewidth}{0.4pt}
    %\hrule
\caption{Runtime comparison of derivative and marked regular
expression matchers for $((a)^{*} + (aa)^{*} + (aaa)^{*} + (aaaa)^{*} + \dots )^{*}$
with input strings of length $n$.} \label{dervsmarGraph}
\end{center}
%\caption{Runtimes of Derivatives vs Marks } \label{RuntimesGraph1}
\end{figure}
Even Antimirov’s partial derivatives, which do not track
POSIX values, may still produce cubic growth in the worst case:
there can be linearly many distinct partial derivatives,
and each derivative can already be quadratic in the size of the
original expression, resulting in an overall cubic upper bound.
\FloatBarrier
\subsubsection{Derivative Extension}
Sulzmann and Lu~\cite{Sulzmann2014} extend Brzozowski’s derivatives to produce 
POSIX values in addition to deciding whether a match exists. 
These values record how the match occurred, for example by showing which part
of the regular expression matched the input string, whether the left or right
branch was taken, and how sequences and Kleene stars were matched.

Sulzmann and Lu provide two variants: a bit-coded construction and an
injection-based construction~\cite{Sulzmann2014}.
In the bit-coded variant, bitsequences encode lexing choices during derivative
construction and, after acceptance, are decoded to the value.
In the injection-based variant, an \textit{inj} function ``injects back'' the
consumed characters into the value; the injection function essentially reverts
the derivative steps to obtain a POSIX value.
We focus here on the bit-coded variant, which more directly inspired our marked
approach.

Bitsequences are sequences over $\{0,1\}$ that encode the choices made in a match.
These bits record branch choices in alternatives and repetitions during matching.
Figure~\ref{BitcodedDerivative} illustrates how bitsequences work with the regular
expression $(a + ab)(b + \ONE)$ on the string \texttt{ab}, proceeding step by step
by constructing derivatives and tracking how the bitsequence grows.
\begin{figure}[ht]

  \begin{center}
    \begin{enumerate}
      \item Step 1: Internalising the regular expression
      \[
      \begin{array}{rcl}
      r' & = & \intern(r) = ({}_0a + {}_1ab) \cdot ({}_0b + {}_1\ONE\,)\\
      \end{array}
      \]
      \item Step 2: input a
      \[
      \begin{array}{rcl}
      \der_a (r') & =           & \der_a(({}_0a + {}_1ab) \cdot ({}_0b + {}_1\ONE\,))\\
                & =           & \der_a({}_0a + {}_1ab) \cdot ({}_0b + {}_1\ONE\,)\\
                & =           & (\der_a({}_0a) + \der_a({}_1ab)) \cdot ({}_0b + {}_1\ONE\,)\\
                & =           & ({}_0\ONE\, + \der_a({}_1a) \cdot b ) \cdot ({}_0b + {}_1\ONE\,)\\
                & = & ({}_0\ONE\, + {}_1\ONE\, \cdot b) \cdot ({}_0b + {}_1\ONE\,)\\
      \end{array}
      \]
      \item Step 3: input b
      \[
      \begin{array}{rcl}
      \der_b(\der_a(r')) & =           & \der_b(({}_0\ONE\, + {}_1\ONE\, \cdot b) \cdot ({}_0b + {}_1\ONE\,))\\
                      & =           & \der_b({}_0\ONE\, + {}_1\ONE\, \cdot b) \cdot ({}_0b + {}_1\ONE\,) \\
                      &             &\quad + \der_b(\, \fuse(\mkeps(r_1),({}_0b + {}_1\ONE\,))) \\
                      & =           & (\der_b({}_0\ONE\,) + \der_b({}_1\ONE\, \cdot b)) \cdot ({}_0b + {}_1\ONE\,) \\
                      &             &\quad + {}_0(\der_b({}_0b) + \der_b({}_1\ONE\,)) \\
                      & =           & (\ZERO\, + \der_b({}_1\ONE\, \cdot b)) \cdot ({}_0b + {}_1\ONE\,) + {}_0({}_0\ONE\, + \ZERO\,) \\
                      & =           & (\der_b({}_1\ONE\,)\cdot b + \der_b({}_1b)) \cdot ({}_0b + {}_1\ONE\,) + {}_0({}_0\ONE\, + \ZERO\,) \\
                      & = & (\ZERO\, \cdot b + {}_1\ONE\,) \cdot ({}_0b + {}_1\ONE\,) + {}_0({}_0\ONE\, + \ZERO\,) \\
      \end{array}
      \]
    \end{enumerate}
  \mbox{}
  \rule{\linewidth}{0.4pt}
  \caption{Step-by-step construction of derivatives with bit-sequences for
$(a + ab) \cdot (b + \ONE)$ with input \texttt{ab}.
Step~1 shows the result of internalization using \fuse,
where alternatives are annotated with bits.
Step~2 takes the derivative with respect to $a$; the two branches of the
result are nullable at this stage: the left branch of $r_1$ and the right branch
of $r_2$ can both match the empty string after consuming the first character,
corresponding to the bitsequences $[0]$ and $[1]$ respectively.
Step~3 takes the derivative with respect to $b$, expanding the expression
due to the nullable concatenation.
Finally, \mkeps extracts the bitsequence $[1,1]$.}
\label{BitcodedDerivative}
  % Bitcoded derivatives of $(a+ab)\cdot (b+\ONE)$ with respect to string \texttt{ba}
  \end{center}
\end{figure}
Initially, the algorithm internalizes the regular expression by applying the
function \intern\ (Figure~\ref{internFunction}), adding bit annotations
to the alternative constructors in the expression: $0$ for left branches and $1$
for right branches~\cite{Sulzmann2014}. This annotation process is implemented by
the function \fuse\ (Figure~\ref{fuseFunction}).
The bitsequence is updated while taking the derivatives to reflect the choices
made during matching.
The function \mkeps\ (Figure~\ref{mkepsFunction}), as defined
by Sulzmann and Lu~\cite{Sulzmann2014}, extracts the bitsequence from nullable
derivatives. A regular expression is nullable if it can match the empty
string.

Derivatives can cause the regular expression to expand. In particular, a
sequence case expands into an alternative when the first part is nullable,
as shown in Figure~\ref{BitcodedDerivative}, Step~3, where the concatenation
becomes nullable. This accounts for the situation in which the first part,
$r_1$, may be skipped during matching: the left branch assumes $r_1$ is not
skipped, while the right branch assumes it is skipped and instead takes the
derivative of $r_2$ directly. This behaviour also illustrates why derivatives
tend to increase in size. Sulzmann and Lu~\cite{Sulzmann2014} use
\fuse(\mkeps($r_1$)) to include the bit annotations (bitsequences) needed
when $r_1$ is skipped; these bits, extracted by \mkeps, indicate how $r_1$
matched the empty string.

After taking the derivatives with respect to the entire input string,
the algorithm checks whether the result is nullable.
If so, it calls \mkeps\ to extract the bitsequence indicating
how the match was obtained.
In Figure~\ref{BitcodedDerivative}, there are two possible matches,
but the algorithm prefers the left one, $[1,1]$, which
encodes the choice of the right branch in the alternative of $r_1$, matching
the string \texttt{ab}, followed by the right branch in the alternative of
$r_2$, which matches the empty string.
Decoding this against the original expression yields the POSIX value
\[
\Seq\,(\,\Right\,(\,\Seq\,(\,a, b\,)), \Right\,(\,\Empty\,))
\]
%intern definition%
\begin{figure}[ht]
  \begin{center}
  \renewcommand{\arraystretch}{1.5}
  \begin{tabular}{lcl}
    $ \intern(\ZERO\,) $       & \dn & $ \ZERO\, $ \\
    $ \intern(\ONE\,) $        & \dn & $ \ONE\, $ \\
    $ \intern(c) $             & \dn & $ c $ \\
    $ \intern(r_1 + r_2) $     & \dn & $ \fuse(\,[0],\, \intern(r_1)) \;+\; \fuse(\,[1],\, \intern(r_2)) $ \\
    $ \intern(r_1 \cdot r_2) $ & \dn & $ \intern(r_1) \cdot \intern(r_2) $ \\
    $ \intern(r^{*}) $           & \dn & $ (\intern(r))^{*} $ \\
  \end{tabular}%
  \renewcommand{\arraystretch}{1.0}
    \mbox{}
    \rule{\linewidth}{0.4pt}
   \caption{The \intern\ function, which internalises regular expressions by
adding bit annotations to alternatives, as defined by
Sulzmann and Lu~\cite{Sulzmann2014}.}
\label{internFunction}
  \end{center}
\end{figure}
%fuse definition%
\begin{figure}[ht]
  \begin{center}
    \renewcommand{\arraystretch}{1.5}
    \begin{tabular}{lcl}
      $ \fuse\; bs\; (c\,_{bs'}) $                & \dn & $ c\,_{(bs\, \At\, bs')} $ \\
      $ \fuse\; bs\; ((r_1 + r_2)\,_{bs'}) $      & \dn & $ (r_1 + r_2)\,_{(bs\, \At\, bs')} $ \\
      $ \fuse\; bs\; ((r_1 \cdot r_2)\,_{bs'}) $  & \dn & $ (r_1 \cdot r_2)\,_{(bs\, \At\, bs')} $ \\
      $ \fuse\; bs\; (r^{*}\,_{bs'}) $              & \dn & $ (r^{*})\,_{(bs\, \At\, bs')} $ \\
      $ \fuse\; bs\; (r^n\,_{bs'}) $              & \dn & $ (r^n)\,_{(bs\, \At\, bs')} $ \\
    \end{tabular}
    \renewcommand{\arraystretch}{1.0}

    \mbox{}
    \rule{\linewidth}{0.4pt}
        \caption{The \fuse\ function, which adds bit annotations to expressions,
as defined by Sulzmann and Lu~\cite{Sulzmann2014}.}
\label{fuseFunction}
  \end{center}
\end{figure}
%mkeps 1 definition%
\begin{figure}[ht]
  \begin{center}
    \renewcommand{\arraystretch}{1.5}
    \begin{tabular}{lcl}
      $ \mkeps(\ONE\,_{bs}) $                & \dn & $ bs $ \\
      $ \mkeps((r_1 + r_2)\,_{bs}) $         & \dn &
      $ \begin{cases}
      bs\, \At\, \mkeps(r_1) & \text{if } \nullable(r_1) \\
      bs\, \At\, \mkeps(r_2) & \text{otherwise}
      \end{cases} $ \\
      $ \mkeps((r_1 \cdot r_2)\,_{bs}) $     & \dn & $ bs\, \At\, \mkeps(r_1)\, \At\, \mkeps(r_2) $ \\
      $ \mkeps((r^{*})\,_{bs}) $               & \dn & $ bs \, \At\,  [1] $ \\
    \end{tabular}
    \renewcommand{\arraystretch}{1.0}

    \mbox{}
    \rule{\linewidth}{0.4pt}
    \caption{The \mkeps\ function, which extracts bit annotations from nullable
expressions, as defined by Sulzmann and Lu~\cite{Sulzmann2014}.}
\label{mkepsFunction}
  \end{center}
\end{figure}

\FloatBarrier
\subsection{Marked Approach}
The marked approach is a method for regular expression matching that tracks
matching progress by inserting and moving marks into the expression.
As noted by Nipkow and Traytel~\cite{NipkowTraytel2014}, the idea can be traced
to the earlier work, in particular to Yamada and Glushkov, who identify positions
in regular expressions by marking their atoms.
Nipkow and Traytel cite Fischer et al.,~\cite{Fischer2010} and Asperti
et al.~\cite{Asperti2010} as reviving this work and developing it in a modern
setting.

This approach aims to enable more efficient matching, particularly in complex
expressions such as $(a^{*} \cdot b^{*})^{*}$, in large alternations like
$(a+b+c+\dots+z)^{*}$, or in nested choices such as
$((a)^{*} + (aa)^{*} + (aaa)^{*} + (aaaa)^{*} + (aaaaa)^{*})^{*}$.The regular 
expression itself does not increase in size; rather, progress is
tracked through the placement of marks. With each input character, 
these marks traverse the regular expression according to a set of rules.
At the end of the input, the expression is evaluated to determine whether the
marks are in positions that make the expression ``final''--that is, whether the
expression accepts the string. 

There is a slight difference in how marks are interpreted in the works of
Fischer et~al.~\cite{Fischer2010} and Asperti et~al.~\cite{Asperti2010}.
In Fischer et~al.’s approach, marks are inserted after a character has been
matched, thereby recording the matched character or subexpression.
In contrast, Asperti et~al.\ interpret the positions of marks as indicating the
potential to match a character: as input is consumed, the marks move through
the regular expression to indicate the next character that can be matched.
In both approaches, acceptance is determined from the final marked structure
of the expression. For Fischer et~al., this corresponds to having matched
characters, whereas for Asperti et~al.\ it requires that the marks end in
positions where the expression can accept the empty string, ensuring that the
entire input has been consumed. If the marks do not reach such positions,
some characters remain unmatched and the matching problem fails.

Nipkow and Traytel, in their work on regular expression equivalence,
build a unified framework for decision procedures on regular expressions.
They describe McNaughton--Yamada--Glushkov and Fischer’s work as
mark-after-atom, meaning that when a character is read, the mark is placed
immediately after it. In their framework, this is expressed by two operations
on marks, namely \follow\ and \readF.
The function \follow\ is, as they describe, similar to an epsilon-closure,
in that it moves all marks in a regular expression to the next atom to be read,
and then \readF\ marks all atoms matching the current input symbol.
The main transition function is therefore written as
\[
\shift\ m\ r\ x = \readF\ x\ (\follow\ m\ r).
\] 

By contrast, Nipkow and Traytel show that the work of Asperti et~al.,
although presented as McNaughton--Yamada, is in fact a dual construction:
mark-before-atom. While Asperti et~al.\ described their approach as following the
McNaughton--Yamada framework, their work actually corresponds
to the opposite variant, where marks are placed before reading the next
atom. In this version, the marks indicate atoms that are about to be read rather
than atoms that have just been consumed.
The main transition function is therefore the reverse of the previous one:
\[
\move\ c\ r\ m = \follow\ m\ (\readF\ c\ r).
\]

\subsubsection{Motivation for a Marked Approach}
The marked approach offers a promising alternative to
derivatives for regular expression matching by addressing some of the
limitations found in the derivative method.
The derivative method provides a more direct way of matching compared to
constructing an NFA and then a DFA, which can suffer from exponential
state explosion, up to $2^n$.
Engines that construct DFAs, such as those used in Rust, attempt
to mitigate this problem by building the automaton dynamically: they
start from the initial state and construct additional DFA states only
as required by the input string.
This avoids generating the full automaton up front, but even so, it can
still lead to large memory usage for expressions such as $r^n$, which
can effectively require $n$ copies of the automaton in the worst case.
Other engines, such as those in Python, use NFAs instead of DFAs.
This design allows for additional features, such as backreferences and
lookarounds, but may result in catastrophic backtracking when the engine
needs to explore a large portion of the automaton---or even the entire
automaton---before finding a match or rejecting an input.
The derivative approach, albeit elegant in design and supported by formal
proofs---unlike many existing engines that lack such guarantees---can still
grow quickly in size, as each derivative may add new subexpressions at
each step.
Even with aggressive simplifications, these subexpressions
cannot always be simplified when they occur at different levels of the
expression tree.

The marked approach relies on propagating marks within the regular expression
rather than constructing new subexpressions. Our main motivation is that this
method could support efficient and high-performance matching with POSIX value
extraction, since it handles matching in a way that avoids some of the
limitations of the derivative-based approach. Inspired by the works of
Fischer et~al.\ and Asperti et~al.~\cite{Fischer2010, Asperti2010}, we aim to
extend the marked approach to extract POSIX values, as well as to handle
extended constructors such as bounded repetitions and intersections.
Our work is primarily based on the algorithm described by
Fischer et~al.~\cite{Fischer2010}. As shown by
Nipkow and Traytel~\cite{NipkowTraytel2014}, the pre-mark algorithm of
Asperti et~al.\ is a dual construction to the post-mark algorithm of
Fischer et~al.; moreover, they prove that the mark-before-atom automaton
is a quotient of the mark-after-atom automaton. Consequently, we adopt
Fischer et~al.’s approach as the foundation for extending the marked
approach to matching with POSIX value extraction.

\section{Our Approach}
We began our work by implementing the marked approach described by Fischer
et~al.~\cite{Fischer2010} in Scala; details are provided in
Appendix~\ref{sec:scala-fischer}. This initial implementation of the algorithm
included only acceptance checking, without any value construction. Over the course
of this work, we developed several versions of the algorithm, each addressing
specific challenges.

The first version extended the marked approach with bit annotations, producing
values but not always the POSIX value. This occurs when marks are overwritten
after a character is matched more than once; further details are provided in
Appendix~\ref{sec:Bit-Annotated1}.

The second version was developed to address two main issues: the overwriting of
marks and the absence of a mechanism to order them so as to always produce the
POSIX value. Initially, we modified the previous version to accumulate all possible
paths to a match, retaining every bitsequence that could lead to acceptance.
Through reasoning and testing, we refined it to the point where we are fairly
confident that it can produce all possible values for a given string and regular
expression, including the POSIX value. One resulting difficulty was the
proliferation of matches in the \Star\ case, where every possible way of matching
was generated. Another difficulty, which also existed in the first version, was
the absence of an ordering for marks during the shifting process. We implemented
an ordering after shifting to evaluate the results of the algorithm, following the
work of Okui and Suzuki~\cite{OkuiSuzuki2013}, although this ordering is not
embedded within the shifting process itself. Instead, all possible values are
generated first, and then the ordering function is applied to select the POSIX
value; further details are provided in
Appendix~\ref{sec:Bit-Annotated2}.

This line of thought eventually led us to the third version of the algorithm,
which uses string-annotated marks. In this version, each mark carries the suffix
of the input string that is still to be matched. This allows us to compare marks
based on their remaining suffixes. We also began annotating these marks with
bitsequences to record the choices made during matching. Although this version is
not yet complete, the initial results are promising. The remainder of this
section describes the third (latest) version; the earlier versions are
presented in the Appendix.

\subsection{String-Carrying Marks - Matcher}
\label{shiftsMatcher}
We initially followed the approach of Fischer et~al.~\cite{Fischer2010} and
Asperti et~al.~\cite{Asperti2010}, in which shifting is performed character by
character and marks are propagated at each step. However, as in our previous
versions, this method made it difficult to maintain an ordering on marks and
then to extract the POSIX value.

To address this, we instead let each mark carry the suffix of the input string
that is still to be matched, starting with an initial mark carrying the full
input string. This gives a more straightforward way of ordering the marks,
based on the remaining suffixes. It also appears to offer a closer correspondence
with derivatives, in that each mark could be associated with subexpressions
created by the derivative, though this point is not yet fully established.
The trade-off is that more information must be stored.
However, with string-carrying marks, we currently have a better handle on
extracting POSIX values.

We define \shifts\ in place of \shift, as the function now operates on the
full input string rather than character by character, as in the previous versions.
The full definition of \shifts\ is given in Figure~\ref{shiftsFunction}.
As discussed in Appendix~\ref{sec:scala-fischer}, the empty-string expression
\ONE\ is not marked. This choice ensures
termination of the \shifts\ function, particularly in the \Star\ case: if
\ONE\ were allowed to produce marks, it would lead to infinite unfolding
and non-termination.

The shifting process starts with a list of marks containing one initial mark
carrying the full input string. As the mark is shifted through the expression,
new marks may be produced depending on its structure,
such as in sequences, stars, or alternatives. For example, in the case of an
alternative, the mark is passed to each branch, and the resulting lists of
marks are combined. Each time a character matches, it is stripped from the
corresponding string carried by the mark. If a character does not match the
prefix of the string carried by a mark, that mark is deleted.
A match occurs at the end of shifting when an empty mark has been produced,
indicating that all characters have been matched. When the input string itself
is empty, \shifts\ is not applied; instead, the \nullable\ function
(with the same definition as in Section~\ref{sec:scala-fischer}, shown in
Figure~\ref{nullableFunction}) is used to check whether the regular expression
accepts the empty string.

The number of marks when using lists is, in most cases, linear in the input
length. The exception is the \Star\ case, where the total number of marks can
grow exponentially or more. This proliferation arises when the inner $r$
contains alternatives and sequences that may add or combine lists of marks,
which are then fed back into the \Star\ through repeated unfoldings. Even though
the length of individual marks becomes shorter with each successful shift, their
number still grows in this way. In such cases, if the regular expression has
$m$ branch choices, then on an input of length $n$, the number of marks can reach
$\bigO{m^{n}}$. For example, for the regular expression $(a + a)^{*}$ and
an input of the form $\underbrace{a \cdots a}_{n}$, we have $m = 2$. For $n = 3$,
the first three unfoldings are as follows, where \At\ stands for list concatenation. 
\[
\begin{aligned}
\text{step 1}:&\quad [\Marked{aa}]\; \At\; [\Marked{aa}] \quad (2) \\
\text{step 2}:&\quad
  [\Marked{a}\; ,\; \Marked{a}]\; \At\; [\Marked{a}\; ,\; \Marked{a}] \quad (4) \\
\text{step 3}:&\quad
  \underbrace{[\Marked{\emptylist}\; ,\; \cdots\;] \At\;[\Marked{\emptylist}\;, \cdots\;]}_{8\ \text{marks}}
\end{aligned}
\]
Using sets, on the other hand, removes duplicates, which are the reason lists
can grow exponentially (or more). The number of distinct marks therefore
remains linear in the input string length. In the same example, we obtain only
$\{\Marked{aa} \rightarrow \Marked{a} \rightarrow \Marked{\emptylist}\}$.
When marks are represented only by their remaining suffixes, there can be at most
$n$ distinct marks for an input string of length $n$, since each mark corresponds
to a unique suffix of the input and there are at most $n$ such suffixes.
Even in the presence of nested stars, the outer unfolding reuses the same set of
suffix marks generated by the inner star, so although the number of recursive
calls can increase, the number of distinct marks at each stage remains bounded
by $n$.

We started with a lexer implementation of this version, and from initial
testing, it produces the POSIX value, yet this remains to be proved.
In that version, we introduce reshuffling, where marks are ordered based on
their remaining strings. At present, the reordering is carried out in the
sequence case after shifting through the first part, and in the \Star\ and $r^n$
cases after shifting into the inner regular expression.

The behaviour of each case of the \shifts\ function is discussed in detail below.
\begin{figure}[ht]
  \begin{center}
  \renewcommand{\arraystretch}{1.7}
  \begin{tabular}{ccl}
    $\shifts(ms, 0)$ & \dn & $\emptylist$ \\
    $\shifts(ms, 1)$ & \dn & $\emptylist$ \\
    $\shifts(ms, d)$ & \dn & $[\, \Marked{s} \mid \Marked{d::s} \in ms \,]$ \\
    $\shifts(ms, r_1 + r_2)$ & \dn & $\shifts(ms, r_1) \; @ \; \shifts(ms, r_2)$ \\

    $\shifts(ms, r_1 \cdot r_2)$ & \dn & $\text{let } ms' = \shifts(ms, r_1) \text{ in}$ \\
                          \multicolumn{3}{@{\hspace{10mm}}l}{%
                          $ \begin{cases}
                          \shifts(ms' \,@\, ms, r_2) \; @ \; ms' & \text{if } \nullable(r_1) \land \nullable(r_2), \\
                          \shifts(ms' \,@\, ms, r_2)             & \text{if } \nullable(r_1), \\
                          \shifts(ms', r_2) \; @ \; ms'      & \text{if } \nullable(r_2), \\
                          \shifts(ms', r_2)                  & \text{otherwise}
                          \end{cases} $} \\
    $\shifts(ms, r^{*})$ & \dn & $\text{let } ms' = \shifts(ms, r) \text{ in}$ \\
                          &&$\text{if } ms' = \emptylist \;\; \text{then } \emptylist \;\; \text{else } \shifts(ms', r^{*}) \; @ \; ms'$ \\
  \end{tabular}
  \renewcommand{\arraystretch}{1.0}

  \mbox{}
  \rule{\linewidth}{0.4pt}
 \caption{The \shifts\ function. Marks are annotated with strings. A mark is represented
as $\Marked{s}$, where $s$ is a string, and $ms$ is a list of
such marks.}
\label{shiftsFunction}
  \end{center}
\end{figure}
%
\paragraph*{\textbf{Character case} $(c)$:}
\begin{itemize}
  \item If the head of the string in a mark matches $c$, it is stripped and
the remainder retained; otherwise, the mark is dropped.
\end{itemize}
\paragraph*{\textbf{Alternative case} $(r_1 + r_2)$:}
\begin{itemize}
  \item The list of marks $ms$ is shifted into both subexpressions $r_1$ and $r_2$.
The two resulting lists are then concatenated, corresponding to matching
with either $r_1$ or $r_2$.
\end{itemize}
\paragraph*{\textbf{Sequence case} $(r_1 \cdot r_2)$:}
\begin{itemize}
\item This is the most elaborate case. It begins by shifting the marks into
the first part of the sequence, $r_1$, producing a new list of marks $ms'$.
The result then depends on several conditions, namely:
    \begin{enumerate}
      \item $\nullable(r_1) \land \nullable(r_2)$: both $r_1$ and $r_2$ are nullable,
            so both can be skipped, and the shifting must account for this by
            combining the results appropriately. To account for $r_1$ being
            skipped, the original marks $ms$ are appended to $ms'$ and then
            shifted into $r_2$. In this way, skipping $r_1$ has the same
            effect as shifting $ms$ directly into $r_2$.
            Finally, $ms'$ itself is appended to the result to account for
            skipping $r_2$, which corresponds to matching with $r_1$ only.

      \item $\nullable(r_1)$: if $r_1$ is nullable, it may be skipped.
            In this case, the original marks $ms$ are appended to $ms'$,
            and the result is then shifted into $r_2$,
            which corresponds to matching directly with $r_2$.

      \item $\nullable(r_2)$: if $r_2$ is nullable, it may be skipped.
            Here, $ms'$ is shifted into $r_2$, and the result is then
            appended with $ms'$ to account for the case where $r_2$ is skipped,
            which corresponds to matching with $r_1$ only.

      \item $\neg\nullable(r_1) \land \neg\nullable(r_2)$:
            if neither $r_1$ nor $r_2$ is nullable,
            the marks $ms'$ are shifted directly into $r_2$.
    \end{enumerate}
\end{itemize}
\paragraph*{\textbf{Star case} $(r^{*})$:}
\begin{itemize}
  \item The list of marks $ms$ is first shifted into $r$, producing a new list $ms'$.
        If $ms'$ is empty, this corresponds to the end of the iterations of the \Star,
        and the result is the empty list.
        Otherwise, $ms'$ records one completed iteration and is appended to
        the result of shifting $ms'$ into the \Star, corresponding to adding a
        new iteration.
\end{itemize}
The matcher is defined as follows.
\[
\matcher(s, r) \; \dn\;
  \begin{cases}
    \nullable(r) & \text{if $s = \emptylist$}, \\
    \Marked{[]} \in \shifts([\Marked{s}], r) & \text{otherwise.}
  \end{cases}
\]

To illustrate the behaviour of this version, we consider three examples.
The first uses the regular expression $(a+(a\cdot c))$ and the string \texttt{ac}.
We show, step by step, how the marks are reduced as characters are stripped and
non-matching paths are dropped. The second example considers the \Star\ case,
highlighting how the marks are unfolded. A third example, $(a^{*}\cdot a^{*})$
with the string \texttt{aa}, shows that, in this version, all possible
marks must be generated, since one of them may be the only one required
to complete the match when the \Star\ expression is part of a larger expression.

In Example~1 (Figure~\ref{StringShiftsExample1}), we consider the regular
expression $(a+(a\cdot c))$ and the string \texttt{ac}. Matching begins
with the initial mark carrying the full string, which is then propagated
into the two subexpressions: the left subexpression $a$ and the right
subexpression $(a\cdot c)$. In the left subexpression $a$, the initial
character matches, so it is stripped from the mark, leaving the string
\texttt{c}. In the right subexpression $(a\cdot c)$, the sequence matches
both characters, so the string in the mark is reduced to the empty string.
At this point, the list of marks $[\Marked{_{[c]}}, \Marked{_{\emptylist}}]$
is obtained, and with no further calls to \shifts\ and one mark reduced to the
empty string, the expression is accepted.

In Example~2 (Figure~\ref{StringShiftsExample2}), we consider the regular
expression $(a^{*})$ and the string \texttt{aaa}. For the \Star\ case,
the process starts by shifting the received marks into the inner expression;
here, the initial mark is shifted into $a$. When the character $a$ matches,
the leading $a$ is stripped from the string carried by the mark, and the inner
\shifts\ call produces a new mark with this shorter string. If the result is
empty—meaning that no character was consumed—no further iteration takes place.
This mechanism has two effects. First, the strings carried by the marks either
remain unchanged or become shorter, in the sense that their length decreases.
Second, it enumerates all the ways the \Star\ expression can match the input,
from no consumption to consuming the entire string. The algorithm therefore
returns a list of marks representing these alternatives, and, as Example~3
will show, one of these marks may be the only one required to complete
the match when the \Star\ expression is part of a larger expression.
This observation also suggests a possible optimisation: rather than
generating all marks, it may be sufficient to generate only those needed
in a given case.

In Example~3 (Figure~\ref{StringShiftsExample3}), we consider the regular expression
$((a^{*}\cdot a)\cdot a)$ and the string \texttt{aaa}. The full string is initially
carried by the mark, as before. The mark is then passed to the first part of the
outer sequence, which is itself a sequence. In step~2, the mark enters the first
part of the inner sequence, $r_1$, which is $a^{*}$, and produces the list of marks
$[\Marked{aa}, \Marked{a}, \Marked_{\emptylist}]$, as shown in Example~2. However,
since $r_1$ is a \Star\ and thus nullable, the original mark $[\Marked{aaa}]$
is appended to the result list of the first part to account for the possibility
of skipping $r_1$. This list of marks is then passed, in step~3, to the second
part of the inner sequence, resulting in the consumption of one character
from each of the marks (and dropping the empty-list mark from $r_1$, since it
cannot match the character in $r_2$). In step~4, the resulting list is passed
to the second part of the outer sequence, consuming another character from the
remaining marks (and again dropping the empty-list mark), resulting in the
final list of marks $[\Marked_{\emptylist}, \Marked{a}]$. 

This example shows that the \Star\ part of the expression produces multiple
marks, and one of them is necessary to complete the match for the entire
expression—in particular, the mark $\Marked{aa}$, which represents consuming
only one character by the \Star\ constructor.
\begin{figure}[ht]
  $\bullet\;$ String \texttt{ac}, regular expression: $(a + (a \cdot c))$
  \begin{center}
    \begin{enumerate}
      \item $\bigl|_{[\Marked{ac}]} \;( a + (a \cdot c) )$
      \item $(\bigl|_{[\Marked{ac}]} a + \bigl|_{[\Marked{ac}]} (a \cdot c))$
      \item $(a \;\bigl|_{[\Marked{c}]} + (a \cdot c)\;\bigl|_{[\Marked_{\emptylist}]})$
    \end{enumerate}
  \end{center}
  \mbox{}
  \hrule
\caption{Example~1: String-carrying shifts for $(a+(a\cdot c))$ with the string \texttt{ac}.}
  \label{StringShiftsExample1}
\end{figure}

\begin{figure}[ht]
  $\bullet\;$ String \texttt{aaa}, regular expression: $(a^{*})$
  \begin{center}
    \begin{enumerate}
      \item $\bigl|_{[\Marked{aaa}]} \;(a^{*})$
      \item $(\bigl|_{[\Marked{aa}]} a^{*})$
      \item $( \bigl|_{[\Marked{aa}, \Marked{a}]} a^{*})$
      \item $(a^{*} \;\bigl|_{[\Marked{aa}, \Marked{a}, \Marked_{\emptylist}]})$
    \end{enumerate}
  \end{center}

  \mbox{}
  \hrule
\caption{Example~2: String-carrying shifts for $(a^{*})$ with the string \texttt{aaa}.}
  \label{StringShiftsExample2}
\end{figure}

\begin{figure}[ht]
  $\bullet\;$ String \texttt{aaa}, regular expression: $((a^{*} \cdot a) \cdot a)$ %(%("a") ~ "a") ~ "a"
  \begin{center}
    \begin{enumerate}
      \item $\bigl|_{[\Marked{aaa}]} \;((a^{*} \cdot a) \cdot a)$
      \item $ \;((a^{*}\, \bigl|_{[\Marked{aa}, \Marked{a}, \Marked_{\emptylist},\Marked{aaa}]}\, \cdot a) \cdot a)$
      \item $ \;((a^{*}\, \cdot a \, \bigl|_{[\Marked{a}, \Marked_{\emptylist},\Marked{aa}]}\,) \cdot a)$
      \item $ \;((a^{*}\, \cdot a  ) \cdot a)\, \bigl|_{[\Marked_{\emptylist},\Marked{a}]}$
    \end{enumerate}
  \end{center}
  \mbox{}
  \hrule
\caption{Example~3: String-carrying shifts for $((a^{*}\cdot a)\cdot a)$ with the string \texttt{aaa}.}
  \label{StringShiftsExample3}
\end{figure}

\FloatBarrier
\section{Future Work}
The development of the marked matcher is still ongoing, and several directions for
future work have been identified:
\begin{itemize}
  \item \textbf{Continue the latest version to produce POSIX values.}  
    Work has started on POSIX value extraction for the latest version of the matcher.
    Initial results indicate that it produces POSIX values.
    Further work is needed to fix the handling of exact repetition $r^n$.
    
  \item \textbf{Extend support for additional operators.}  
    Future work will focus on extending the matcher to handle intersection and
    negation operators. This will require extending both the
    shifting function and the bitcoding scheme so that these operators are
    handled correctly during matching.
    
  \item \textbf{Formal verification in Isabelle/HOL.}  
    Work has started on formally proving the correctness of the latest version
    of the algorithm in Isabelle/HOL. The next step is to prove that it always
    produces the POSIX value.
    
  \item \textbf{Performance and optimisation.}  
    We are exploring ways to make the algorithm more efficient. We plan to
    investigate strategies to control or reduce mark proliferation in complex
    expressions and to handle bitcode annotations more efficiently.
    
  \item \textbf{Parsing with marks.}  
    We plan to explore extending the marked approach to parsing by generalising from
    regular expressions to context-free grammars, in analogy to
    Might et~al.’s~\cite{Might2011} generalisation of derivatives to obtain
    a functional parsing algorithm.
\end{itemize}


\bibliographystyle{abbrv}
\bibliography{urules}
\addcontentsline{toc}{section}{References}

\FloatBarrier
\appendix
\section{Appendix}

\subsection{Scala Implementation of Fischer et~al.’s Marked Approach}
\label{sec:scala-fischer}

In Fischer et~al.’s approach, the marks are shifted through the regular expression
with each input character. The process starts with an initial mark inserted at the
beginning, which is then moved step by step as the input is consumed.
This behaviour is implemented by the function \shift, which performs the core logic
of the algorithm. The initial specification of this function is given below, as we
have developed several versions throughout our work.

The following presents the Scala implementation of the shifting behaviour as originally
defined by Fischer et~al.~\cite{Fischer2010}.
The \shift\ function takes as input a regular expression to match against, a flag $m$,
and a character $c$, and returns a \emph{marked regular expression}—that is, a
regular expression annotated with marks. We write a marked regular expression as
$\bullet\,r$, where the preceding dot indicates that the expression $r$ has been
annotated with marks to record the progress of matching.
%\[ \shift(m, c, r) \to \Marked{r}\]
\[
\renewcommand{\arraystretch}{1.5}
\shift(m, c, r) =
\begin{array}{l}
  \begin{cases}
    \Marked{r} & \text{if $c$ matches in $r$}, \\
    \ZERO\, & \text{otherwise.}
  \end{cases}
\end{array}
\renewcommand{\arraystretch}{1.0}
\]

The flag $m$ indicates the mode of operation: when set to true, a new mark is
introduced; otherwise, the function shifts the existing marks.
This was realised in our first implementation by adding a Boolean attribute to the
character constructor to represent a marked character.
In later versions, we instead introduced a wrapper constructor around the character
constructor to explicitly represent a marked character.
%shift-fischer definition%
\begin{figure}[ht]
  \begin{center}
    \renewcommand{\arraystretch}{1.5}
    \begin{tabular}{lcl}
      \shift$(m,c,\ZERO)$ & \dn & $\ZERO$ \\
      \shift$(m,c,\ONE)$  & \dn & $\ONE$ \\
      \shift$(m,c,d)$     & \dn &
                                  $\begin{cases}
                                  \bullet d & \text{if } c=d \land m \\
                                  d         & \text{otherwise}
                                  \end{cases}$ \\

      \shift$(m,c,r_1 + r_2)$ & \dn & $shift(m,c,r_1) + shift(m,c,r_2)$ \\
      \shift$(m,c,r_1 \cdot r_2)$ & \dn & \medskip \\
                                    \multicolumn{3}{@{\hspace{10mm}}l}{
                                    $\begin{cases}
                                    shift(m,c,r_1) \cdot shift(true,c,r_2)  & \text{if } m \land \nullable(r_1) \\
                                    shift(m,c,r_1) \cdot shift(true,c,r_2)  & \text{if } \fin(r_1) \\
                                    shift(m,c,r_1) \cdot shift(false,c,r_2) & \text{otherwise}\\
                                    \end{cases}$} \\
      \shift$(m,c,r^{*})$ & \dn &
                          $\begin{cases}
                          shift(true,c,r^{*}) & \text{if } \fin(r) \\
                          shift(m,c,r^{*})    &  \\
                          \end{cases}$ \\
    \end{tabular}
    \renewcommand{\arraystretch}{1.0}

    \mbox{}
    \rule{\linewidth}{0.4pt}
   \caption{The \shift\ function in the Scala implementation of 
   Fischer’s Marked Approach.}\label{shiftFunction}
  \end{center}
\end{figure}
\noindent
Shifting marks for the base cases $\ZERO$ and $\ONE$ is straightforward:
$\ZERO$ cannot be marked, and $\ONE$—for now—will not carry a mark,
since it matches only the empty string. In the initial algorithm, $\ONE$ was
not marked, and this choice is carried over into later versions. The reason is
to avoid complications and ensure termination of the \shift\ function—some of which
become apparent in the later versions, as discussed in subsequent sections.\footnote{%
Our choice follows Fischer et~al.~\cite{Fischer2010}, where $\ONE$ is left unmarked
(`shift EPS = EPS`). Asperti et~al.~\cite{Asperti2010}, on the other hand,
use pointed regular expressions (pREs), where acceptance of the empty string
is represented by the trailing point being set to true.}
The behaviour of the remaining cases is described next.
\begin{itemize}
  \item \textbf{Character case} $(d)$:  
  If the input character $c$ matches $d$ and the flag $m$ is true,
  a mark is added and stored in the character constructor; otherwise, the character
  remains unmarked.  

  \item \textbf{Alternative case} $(r_1 + r_2)$:  
  Marks are shifted into both subexpressions, since either branch may match the input
  character.  

  \item \textbf{Sequence case} $(r_1 \cdot r_2)$:  
  \begin{itemize}
    \item If $r_1$ is neither nullable nor in a final position, marks are shifted only
    into $r_1$, indicating that matching proceeds with the first component. 

    \item If $r_1$ is nullable and may be skipped, marks are shifted into both $r_1$
    and $r_2$, so that either component can begin matching. 

    \item If $\fin(r_1)$ holds, meaning $r_1$ has finished matching, marks are shifted
    into $r_2$ to continue matching with that component.  
  \end{itemize}

  \item \textbf{Star case} $(r^{*})$:  
  Marks are shifted into the subexpression if $m$ is true or if $\fin(r)$ holds.  
\end{itemize}

The formal definitions of the auxiliary functions $\fin$ and $\nullable$ appear
in Figures~\ref{finFunction} and~\ref{nullableFunction}, as defined by Fischer
et~al.~\cite{Fischer2010}.
%fin definition%
\begin{figure}[ht]
  \begin{center}
    \renewcommand{\arraystretch}{1.5}
    \begin{tabular}{lcl}
      $\fin(\ZERO\,)$            & \dn & $\text{false}$ \\
      $\fin(\ONE\,)$             & \dn & $\text{false}$ \\
      $\fin(c)$                  & \dn & $\text{false}$ \\
      $\fin(\Marked{c})$         & \dn & $\text{true}$ \\
      $\fin(r_1 + r_2)$          & \dn & $\fin(r_1) \lor \fin(r_2)$ \\ 
      $\fin(r_1 \cdot r_2)$      & \dn & $(\fin(r_1) \land \nullable(r_2)) \lor \fin(r_2)$ \\
      $\fin(r^{*})$                & \dn & $\fin(r)$  
    \end{tabular}
    \renewcommand{\arraystretch}{1.0}

    \mbox{}
    \rule{\linewidth}{0.4pt}
\caption{The \fin\ function, which checks whether a marked expression is in a
final state, as defined by Fischer et~al.~\cite{Fischer2010}.}\label{finFunction}
  \end{center}
\end{figure}

%nullable definition%
\begin{figure}[ht]
  \begin{center}
    \renewcommand{\arraystretch}{1.5}
    \begin{tabular}{lcl}
      $\nullable(\ZERO\,)$         & \dn & $\text{false}$ \\
      $\nullable(\ONE\,)$          & \dn & $\text{true}$ \\
      $\nullable(c)$               & \dn & $\text{false}$ \\
      $\nullable(r_1 + r_2)$       & \dn & $\nullable(r_1) \lor \nullable(r_2)$ \\ 
      $\nullable(r_1 \cdot r_2)$   & \dn & $\nullable(r_1) \land \nullable(r_2)$ \\
      $\nullable(r^{*})$             & \dn & $\text{true}$
    \end{tabular}
    \renewcommand{\arraystretch}{1.0}
    
    \mbox{}
    \rule{\linewidth}{0.4pt}
    \caption{The \nullable\ function, which checks whether an expression can match
the empty string, as defined by Fischer et~al.~\cite{Fischer2010}.} \label{nullableFunction}
  \end{center}
\end{figure}

\subsection{Bit-Annotated Version~1}
\label{sec:Bit-Annotated1}
In this version, we extended the marked approach with bit sequencing.
Inspired by Sulzmann and Lu~\cite{Sulzmann2014}, we introduced bitcodes in
the form of lists attached to each mark, which are incrementally built as
the marks are shifted. This version produces a value, though not necessarily
the POSIX value, because when a character is matched more than once at the
same point, the associated bitsequence may be overwritten. This can cause
value erasure and, in some cases, the loss of the POSIX value, as illustrated
in Example~2 below. We use the bit annotations $0$ and $1$, similar to the
bitcoded derivatives described by Sulzmann and Lu~\cite{Sulzmann2014}.  

In contrast to the original work by Fischer et~al., our \shift\ function takes
an additional argument, \Bits.
As \shift\ is applied, bits are appended to the bitsequence. For example, when
shifting through an alternative, $0$ is appended to the bitsequence and
passed to the left subexpression, and $1$ to the right subexpression.
If the input character matches a leaf character node, it is wrapped by the newly
defined \POINT\ constructor, which represents the mark. The associated bitsequence
is stored inside this constructor together with the character.

We define the auxiliary function \mkfin\ to extract the bitsequence representing
the path, in bits, describing how the expression matched. We adapt \mkeps\
from Sulzmann and Lu~\cite{Sulzmann2014} and from Tan and
Urban~\cite{TanAndUrban2023}, who define it to construct a value tree
and a bitsequence, respectively, for how a nullable expression matches the
empty string. Our version also returns a bitsequence: $0$ for the left branch
and $1$ for the right branch in choices; and for the \Star\ case, $0$ indicates
the start of an iteration and $1$ its end, with the single bit $1$ used for the
empty-star case. The auxiliary functions \mkfin\ and \mkeps\ are defined
in Figures~\ref{mkfinFunction} and~\ref{mkepsBit1Function}, while the
definition of \fin\ is the same as in Section~\ref{sec:scala-fischer}.
We define \shift\ for this version as follows:
\[
\renewcommand{\arraystretch}{1.5}
\shift(m, c, bs, r) =
\begin{array}{l}
  \begin{cases}
    \Marked{r\,_{bs'}} & \text{if $c$ matches in $r$}, \\
    \ZERO\,        & \text{otherwise.}
  \end{cases} 

\end{array}
\renewcommand{\arraystretch}{1.0}
\]
where $bs'$ is the updated bitsequence. The complete definition appears
in Figure~\ref{bitAnnotatedShiftFunction}.
The behaviour of each case of the \shift\ function is discussed in detail below.
%shift 1 definition%
\begin{figure}[ht]
  \begin{center}
    \renewcommand{\arraystretch}{1.5}
    \begin{tabular}{lcl}
      $\shift(m,bs,c,\ZERO\,)$   & \dn & $\ZERO$ \\
      $\shift(m,bs,c,\ONE\,)$    & \dn & $\ONE$ \\
      $\shift(m,bs,c,d)$         & \dn &
                                  $\begin{cases}
                                  \Marked{d_{bs}} & \text{if } m \land d = c \\
                                  d               & \text{otherwise}
                                  \end{cases}$ \\
      $\shift(m,bs,c,r_1 + r_2)$ & \dn &
      $\shift(m,\, bs \oplus 0,\, c,\, r_1) \;+\; \shift(m,\, bs \oplus 1,\, c,\, r_2)$ \\
      $\shift(m,bs,c,r_1 \cdot r_2)$ & \dn & \\
                                      \multicolumn{3}{@{\hspace{15mm}}l}{
                                      $\begin{cases}
                                      \shift(m,bs,c,r_1) \cdot \shift(\text{true},\, bs \At \mkeps(r_1),\, c,\, r_2) & \text{if } m \land \nullable(r_1) \\
                                      \shift(m,bs,c,r_1) \cdot \shift(\text{true},\, bs \At \mkfin(r_1),\, c,\, r_2) & \text{if } \fin(r_1) \\
                                      \shift(m,bs,c,r_1) \cdot \shift(\text{false}, \emptylist,\, c,\, r_2)          & \text{otherwise}
                                      \end{cases}$} \\
      $\shift(m,bs,c,r^{*})$ & \dn & \\
                            \multicolumn{3}{@{\hspace{15mm}}l}{
                            $\begin{cases}
                            (\shift(m,\, bs \oplus 0,\, c,\, r))^{*} & \text{if } m \\
                            (\shift(\text{true},\, bs \At (\mkfin(r) \oplus 1),\, c,\, r))^{*} & \text{if } m \land \fin(r) \\
                            (\shift(\text{true},\, \mkfin(r) \oplus 0,\, c,\, r))^{*} & \text{if } \fin(r) \\
                            (\shift(\text{false}, \emptylist,\, c,\, r))^{*} & \text{otherwise}
                            \end{cases}$}
    \end{tabular}
    \renewcommand{\arraystretch}{1.0}

    \mbox{}
    \rule{\linewidth}{0.4pt}
\caption{Bit-Annotated Version~1 \shift. Here, $\oplus$ stands for appending
a bit to a bitsequence, while $\At$ stands for concatenation of two
bitsequences.}
\label{bitAnnotatedShiftFunction}
  \end{center}
\end{figure}
\paragraph*{\textbf{Character case} $(d)$:}
\begin{itemize}
  \item If the input character $c$ matches $d$ and the flag $m$ is true,
  a \POINT\ wraps the constructor $c$, with the updated bitsequence $bs$
  stored inside the \POINT. Otherwise, the character remains unmarked.
\end{itemize}

\paragraph*{\textbf{Alternative case} $(r_1 + r_2)$:}
\begin{itemize}
  \item Marks are shifted as before, and the direction of the match is annotated:
  $0$ is added to the bitsequence passed to the left subexpression,
  and $1$ is added to the bitsequence passed to the right subexpression.
\end{itemize}

\paragraph*{\textbf{Sequence case} $(r_1 \cdot r_2)$:}
\begin{itemize}
  \item If $r_1$ is nullable, a mark is shifted to both $r_1$ and $r_2$:
  $bs$ is passed to $r_1$ (representing the path to this expression),
  and $bs \At \mkeps(r_1)$ is passed to $r_2$, where $\mkeps$ returns the bits
  for an empty-string match. This corresponds to the case where the first part
  of the sequence is skipped.

  \item If $r_1$ is in a final position (that is, it has finished matching),
  a mark is shifted to $r_2$ with the bitsequence describing how $r_1$ was matched,
  extracted using the \mkfin\ function, i.e.\ $bs \At \mkfin(r_1)$.

  \item Otherwise, marks are shifted only into $r_1$, with $bs$ representing
  the current path to $r_1$.
\end{itemize}

\paragraph*{\textbf{Star case} $(r^{*})$:}
\begin{itemize}
  \item If a new mark is introduced, $bs$ is passed with $0$ appended,
  representing the beginning of a new iteration of the star.

  \item If a new mark is introduced and $r$ is in a final position,
  $bs \At \mkfin(r)$ is passed with $1$ appended, combining the bits describing
  the path to $r^{*}$ with the bits showing how $r$ reached a final position.

  \item If $r$ is in a final position, $\mkfin(r)$ is passed with $0$ appended,
  representing the start of a new iteration.

  \item If no new mark is introduced, the existing marks are shifted with an
  empty bitsequence.
\end{itemize}
%The functions \mkfin\ and \mkeps\ are defined below; the definition of \fin\ is the same as in section~\ref{sec:scala-fischer}.  
%mkfin definition%
\begin{figure}[ht]
  \begin{center}
    \renewcommand{\arraystretch}{1.5}
    \begin{tabular}{lcl}
      $\mkfin(\Marked{_{bs}\,r})$            & \dn & $bs$ \\[0.2ex]
      $\mkfin(\Marked{_{bs}\,c})$            & \dn & $bs$ \\
      $\mkfin(r_1 + r_2)$                    & \dn &
        $\begin{cases}
          \mkfin(r_1) & \text{if } \fin(r_1) \\
          \mkfin(r_2) & \text{otherwise}
        \end{cases}$ \\
      $\mkfin(r_1 \cdot r_2)$                & \dn &
        $\begin{cases}
          \mkfin(r_1) \At \mkeps(r_2) & \text{if } \fin(r_1) \land \nullable(r_2) \\
          \mkfin(r_2)                 & \text{otherwise}
        \end{cases}$ \\
      $\mkfin(r^{*})$                          & \dn & $\mkfin(r) \oplus 1$
    \end{tabular}
    \renewcommand{\arraystretch}{1.0}

    \mbox{}
    \rule{\linewidth}{0.4pt}
   \caption{The \mkfin\ function, which extracts the bitsequence describing
how a regular expression matched.}
\label{mkfinFunction}
  \end{center}
\end{figure}
%mkeps definition%
\begin{figure}[ht]
  \begin{center}
    \renewcommand{\arraystretch}{1.5}
    \begin{tabular}{lcl}
      $\mkeps(\ONE)$                       & \dn & $\emptylist$ \\
      $\mkeps(r_1 + r_2)$                    & \dn &
        $\begin{cases}
          0 \oplus \mkeps(r_1) & \text{if } \nullable(r_1) \\
          1 \oplus \mkeps(r_2) & \text{otherwise}
        \end{cases}$ \\
      $\mkeps(r_1 \cdot r_2)$                & \dn & $\mkeps(r_1) \At \mkeps(r_2)$ \\
      $\mkeps(r^{*})$                          & \dn & $[1]$
    \end{tabular}
    \renewcommand{\arraystretch}{1.0}

    \mbox{}
    \rule{\linewidth}{0.4pt}
    \caption{The \mkeps\ function, which returns the bitsequence describing
how a nullable expression matches the empty string.}
\label{mkepsBit1Function}
  \end{center}
\end{figure}

Next, we present two examples of matching a string and extracting a value.
The first example shows how the bitsequence is constructed during matching,
while the second demonstrates a case where the algorithm fails to produce the
POSIX value. When shifting to a point (an already marked character) and the
character matches again, the associated bitsequence is overwritten, which can
lead to value erasure and, in particular, to the loss of the POSIX value,
as illustrated in the second example. 
  \begin{figure}[ht]
    $\bullet\;$ String \texttt{ba}, regular expression: $(a \cdot b + b \cdot a)$
    \[
    \begin{array}{rcl}
      \shift\; b & \rightarrow & (a \cdot b) + (\Marked{_{[1]}\, b} \cdot a)\\\\
      \shift\; a & \rightarrow & (a \cdot b) + (b \cdot \Marked{_{[1]}\, a} )\\
    \end{array}
    \]
    \hrule

\caption{Example~1: Bit-Annotated Version~1 \shift\ for $(a\cdot b + b\cdot a)$
with the string \texttt{ba}.}
\label{BitAnnotatedShiftExample1}

    \mbox{}

  \end{figure}   

In Example~1 (Figure~\ref{BitAnnotatedShiftExample1}), the process starts
by shifting the first character, $b$, into the regular expression. Since the
expression is an alternative, \shift\ is applied to both branches. However,
only the right branch matches, because its first subexpression
begins with $b$. A mark is therefore set with the bitcode list $[1]$.
Shifting the second character, $a$, continues into the concatenation on the
right branch. Here, the first part is already final, as it matched the preceding
$b$, so the shift moves on to the second part, which matches $a$.
With no further calls to \shift, \mkfin\ is applied, because the regular
expression has reached a final position, indicated by a mark at the end of the
expression. The resulting bitsequence is $[1]$, recording that the match
followed the right branch of the alternative.
  \begin{figure}[ht]
    $\bullet\;$ String \texttt{aaa}, regular expression: $(a + a \cdot a)^{*}$
    \[
    \begin{array}{rcl}
      \shift\; a & \rightarrow & ( \Marked{_{[0,0]\,} a} +  \Marked{_{[0,1]}\,a \cdot a} )^{*}\\\\
      \shift\; a & \rightarrow & ( \Marked{_{[0,0,0,0]\,} a} +  \Marked{_{[0,0,0,1]}\,a \cdot \Marked{_{[0,1]}\,a} } )^{*}\\\\ 
      \shift\; a & \rightarrow & ( \Marked{_{[0,0,0,0,0,0]\,} a} +  \Marked{_{[0,0,0,0,0,1]}\,a \cdot \Marked{_{[0,0,0,1]}\,a} } )^{*}\\\\
    \end{array}
    \]

    \mbox{}
    \hrule
    \caption{Example~2: Bit-Annotated Version~1 \shift\ for $(a + a\cdot a)^{*}$
   with the string \texttt{aaa}.}
   \label{BitAnnotatedShiftExample2}
  \end{figure}
In Example~2 (Figure~\ref{BitAnnotatedShiftExample2}), after the first shift
on $a$, a mark is placed on the left branch with bits $[0,0]$, indicating the
start of a \Star\ iteration followed by a left choice. In the right
subexpression, the mark on $r_1$ of the $a \cdot a$ sequence carries bits
$[0,1]$, representing the start of the \Star\ iteration followed by a right
choice. When the second character is shifted, the right subexpression $r_2$
with bits $[0,1]$ is overwritten during the third shift, when those bits should
instead be preserved. These bits correspond to the POSIX match, which begins
by matching the right-hand side first and then performing another iteration to
match the left-hand side. The correct bitsequence in that case would be
$[0,1,0,0]$, with the final $1$ indicating the end of the \Star\ iteration.
This behaviour arises because \POINT\ stores only a single bitsequence at a
time, with no mechanism for preserving multiple marks.

\subsection{Bit-Annotated Version~2}
\label{sec:Bit-Annotated2}
In this version, we updated the previous Bit-Annotated Version~1 to accumulate
all possible bitsequences leading to a match, rather than storing only a
single bitsequence at each marked position. This change was motivated by two
issues observed in Version~1: the overwriting of marks when a character is
matched more than once at the same position—which can cause the POSIX value
to be lost—and the absence of a mechanism to order marks during shifting so
as to always produce the POSIX value.

To address the first issue, each marked position now carries a list of
bitsequences. Whenever a character is matched, instead of overwriting the
existing bitsequence, the new bitsequence is appended to the list already
stored at that position.

To address the second issue, ordering is applied after the shifting process,
using a function inspired by the work of Okui and Suzuki~\cite{OkuiSuzuki2013}.
This ordering is not embedded within shifting itself: all possible values are
first generated and then ordered externally.

The definition of \shift\ for this version is shown in
Figure~\ref{bitAnnotated2Shiftfunction}. The key difference in this version,
compared to Version~1, is that all possible paths are considered. For example,
in the sequence case, the bitsequences that arise when $r_1$ is nullable and can
be skipped are passed along together with those that arise when $r_1$ is in a
final position. This means that the right-hand side of a sequence ($r_2$)
receives both lists of bitsequences—those corresponding to $r_1$ being skipped
and those where $r_1$ is final—ensuring that all possible paths are accounted
for. Similarly, the function \mkfin\ is updated to return all possible
bitsequences.

The auxiliary function \mkfin\ now returns lists of bitsequences, while
\mkeps\ remains unchanged from the previous version. The updated definition
of \mkfin\ is given in Figure~\ref{mkfin2Definition}.

By accumulating lists of bitsequences at each position, no potential path is
lost due to overwriting. Although a formal proof is not yet complete,
extensive testing and reasoning give us confidence that this version
generates all possible values for matching a string against a regular
expression, including the POSIX value. This improvement, however, comes at the
cost of an increase in the number of values returned—particularly in the
\Star\ case, where every possible way of matching is generated.

\begin{figure}[ht]
  \begin{center}
    \renewcommand{\arraystretch}{1.5}
    \begin{tabular}{lcl}
      $\shift(m,bs,c,\ZERO\,)$   & \dn & $\ZERO$ \\
      $\shift(m,bs,c,\ONE\,)$    & \dn & $\ONE$ \\
      $\shift(m,bs,c,d)$         & \dn &
                                  $\begin{cases}
                                  \Marked{d_{bs}} & \text{if } m \land d = c \\
                                  d               & \text{otherwise}
                                  \end{cases}$ \\
      $\shift(m,bs,c,r_1 + r_2)$ & \dn &
      $\shift(m,\, bs \bigoplus 0,\, c,\, r_1) \;+\; \shift(m,\, bs \bigoplus 1,\, c,\, r_2)$ \\
      $\shift(m,bs,c,r_1 \cdot r_2)$ & \dn &  $ \shift(m,bs,c,r_1)\, \cdot$\\
      \multicolumn{3}{@{\hspace{15mm}}l}{
      $\begin{cases}
      \shift(m,\, (bs \At \mkeps(r_1)\, \At\ \mkfin(r_1)),\, c,\, r_2)
        & \text{if } m \land \nullable(r_1) \land \fin(r_1) \\
      \shift(m,\, bs \At \mkeps(r_1),\, c,\, r_2)
        & \text{if } m \land \nullable(r_1) \\
      \shift(\text{true},\, \mkfin(r_1),\, c,\, r_2)
        & \text{if } \fin(r_1) \\
      \shift(\text{false},\, \emptylist,\, c,\, r_2)
        & \text{otherwise}
      \end{cases}$} \\
      $\shift(m,bs,c,r^{*})$ & \dn & \\
      \multicolumn{3}{@{\hspace{15mm}}l}{
      $\begin{cases}
      \big(\shift(m,\, (bs\, \At\ \mkfin(r)) \bigoplus 0,\, c,\, r)\big)^{*}
        & \text{if } m \land \fin(r) \\
      \big(\shift(\text{true},\, \mkfin(r) \bigoplus 0,\, c,\, r)\big)^{*}
        & \text{if } \fin(r) \\
      \big(\shift(m,\, bs \bigoplus 0,\, c,\, r)\big)^{*}
        & \text{if } m \\
      \big(\shift(\text{false},\, \emptylist,\, c,\, r)\big)^{*}
        & \text{otherwise}
      \end{cases}$}
    \end{tabular}
    \renewcommand{\arraystretch}{1.0}

    \mbox{}
    \rule{\linewidth}{0.4pt}
    \caption{Bit-Annotated Version~2 \shift. Here,
  $\bigoplus$ stands for appending a single element to each member of a list,
  $\lessdot$ stands for appending a list to each member of a list,
  and $\At$ stands for list concatenation.}
    \label{bitAnnotated2Shiftfunction}
  \end{center}
\end{figure}
%mkfin2
\begin{figure}[ht]
  \begin{center}
    \renewcommand{\arraystretch}{1.5}
    \begin{tabular}{lcl}
      $\mkfin(\Marked{_{bs}\,c})$            & \dn & $bs$ \\
      $\mkfin(r_1 + r_2)$                    & \dn &
        $\begin{cases}
          \mkfin(r_1) \;\At\; \mkfin(r_2) & \text{if } \fin(r_1) \land \fin(r_2),\\
          \mkfin(r_1)                      & \text{if } \fin(r_1),\\
          \mkfin(r_2)                      & \text{if } \fin(r_2).
        \end{cases}$ \\[1.2ex]
        $\mkfin(r_1 \cdot r_2)$ & \dn & \\
          \multicolumn{3}{@{\hspace{15mm}}l}{$
            \begin{cases}
              \mkfin(r_2) \;\At\; \bigl(\mkfin(r_1) \lessdot \mkeps(r_2)\bigr)
                & \text{if } \fin(r_1) \land \nullable(r_2) \land \fin(r_2),\\
              \mkfin(r_1) \lessdot \mkeps(r_2)
                & \text{if } \fin(r_1) \land \nullable(r_2),\\
              \mkfin(r_2)
                & \text{otherwise.}
            \end{cases}$} \\
      $\mkfin(r^{*})$                          & \dn & $\mkfin(r) \bigoplus\, 1$
    \end{tabular}
    \renewcommand{\arraystretch}{1.0}

    \mbox{}
    \rule{\linewidth}{0.4pt}
    \caption{The \mkfin\ function for Bit-Annotated Version~2, which returns a list of all
bitsequences in a point, representing all possible match paths.}
    \label{mkfin2Definition}
  \end{center}
\end{figure}
%mkeps2

\subsection{String-Carrying Marks — Lexer (so far)}
In this version, we extend the string-carrying-marks design by attaching a bit
sequence to each mark, so that each mark consists of its remaining suffix and its
corresponding bitsequence. Similar to the Bit-Annotated Versions~1 and~2, the bit
sequences are built incrementally as the marks are shifted through the regular
expression. The difference here is that the order of marks is determined by their
remaining suffixes: shorter suffixes are prioritised over longer ones, while marks
with equal suffix length preserve their original order. This reordering, which we
call \reshuffle, is applied after shifting through the first part of a sequence, after
shifting into the inner regular expression of a \Star, and during the handling of
$r^n$. Once shifting is complete, we extract the bitsequence of the first mark whose
carried suffix is empty (i.e., fully consumed).

We define \shifts\ in this version in Figure~\ref{shiftsFunctionLexer}.
The behaviour of each case of the \shifts\ function is discussed in detail below.

\begin{figure}[ht]
  \begin{center}
  \renewcommand{\arraystretch}{1.7}
  \begin{tabular}{ccl}
    $\shifts(ms, 0)$ & \dn & $\emptylist$ \\
    $\shifts(ms, 1)$ & \dn & $\emptylist$ \\
    $\shifts(ms, d)$ & \dn & $[\, \Marked{s} \mid \Marked{d::s} \in ms \,]$ \\

    $\shifts(ms, r_1 + r_2)$ & \dn &
      $\shifts(ms \bigoplus 0, r_1) \; @ \; \shifts(ms \bigoplus 1, r_2)$ \\

    $\shifts(ms, r_1 \cdot r_2)$ & \dn &
      $\text{let } ms' = \shifts(ms, r_1).\reshuffle \text{ in}$ \\
      \multicolumn{3}{@{\hspace{10mm}}l}{%
      $ \begin{cases}
        (\, ms' \lessdot \mkeps(r_2) \,) 
        \; @ \;
        \big[\, \shifts([\Marked{s}], r_2) \mid \Marked{s} \in ms' \,\big]^{@}
        \; @ \;
        \shifts(ms \lessdot \mkeps(r_1), r_2)
        & \\ \text{if } \nullable(r_1) \land \nullable(r_2),\\
        
        \big[\, \shifts([\Marked{s}], r_2) \mid \Marked{s} \in ms' \,\big]^{@}
        \; @ \;
        \shifts(ms \lessdot \mkeps(r_1), r_2)
        & \\ \text{if } \nullable(r_1) \land \neg\nullable(r_2),\\

        (\, ms' \lessdot \mkeps(r_2) \,) \; @ \;
        \big[\, \shifts([\Marked{s}], r_2) \mid \Marked{s} \in ms' \,\big]^{@}
        & \\ \text{if } \neg\nullable(r_1) \land \nullable(r_2),\\

        \big[\, \shifts([\Marked{s}], r_2) \mid \Marked{s} \in ms' \,\big]^{@}
        &\\ \text{otherwise.}
      \end{cases}$} \\

    $\shifts(ms, r^{*})$ & \dn &
      $\text{let } ms' = \shifts(ms \oplus 0, r).\reshuffle \text{ in}$ \\
      &&$\text{if } ms' = \emptylist \;\; \text{then } \emptylist 
      \;\; \text{else } \shifts(ms', r^{*}) \; @ \; (\, ms' \oplus 1 \,)$ \\

    $\shifts(ms, r^{n})$ & \dn &
      $\text{if } n=0 \text{ then } \emptylist$ \\
      && $\text{else if } n=1 \text{ then} \; \shifts(ms \oplus 0, r).\reshuffle$ \\ 
      &&$\text{else let } ms' = \shifts(ms \oplus 0, r).\reshuffle \text{ in}$ \\[2pt]
      &&$\text{if } ms' = \emptylist \text{ then } \emptylist$ \\
      && $\text{ else } 
      \begin{cases}
        (\, ms' \oplus 1 \,) \; @ \; \shifts(ms', r^{n-1})
        & \text{if } \nullable(r),\\
        \shifts(ms', r^{n-1})
        & \text{otherwise.}
      \end{cases}$

  \end{tabular}
  \renewcommand{\arraystretch}{1.0}

  \mbox{}
  \rule{\linewidth}{0.4pt}
\caption{The \shifts\ function for the String-Carrying Marks — Lexer version.
Here, $\bigoplus$ stands for appending a single element to each member of a list,
$\lessdot$ stands for appending a list to each member of a list,
and $\At$ stands for list concatenation.}
  \label{shiftsFunctionLexer}
  \end{center}
\end{figure}
\paragraph*{\textbf{Character case} $(c)$:}
\begin{itemize}
  \item Similar to the previous version described in Section~\ref{shiftsMatcher}.
\end{itemize}

\paragraph*{\textbf{Alternative case} $(r_1 + r_2)$:}
\begin{itemize}
  \item The list of marks $ms$ is shifted into both subexpressions $r_1$ and $r_2$, with $0$
  appended to each mark’s bitsequence in the left-hand side $r_1$, and $1$ appended in $r_2$.
\end{itemize}

\paragraph*{\textbf{Sequence case} $(r_1 \cdot r_2)$:}
\begin{itemize}
  \item As in previous versions, this is the most elaborate case, especially with the ordering,
  which determines how marks are shifted and combined to preserve the order of remaining suffixes. 
  It begins by shifting the marks into the first part of the sequence, $r_1$, producing a new list 
  of marks $ms'$ as before. We then apply $\reshuffle$, which reorders the list of marks based on 
  the remaining suffixes, giving lower priority to those with longer remaining strings. 
  The result then depends on several conditions, namely:
  \begin{enumerate}
    \item $\nullable(r_1) \land \nullable(r_2)$: when both $r_1$ and $r_2$ can be skipped,
    the returned list of marks must preserve the remaining-suffix order.
    There are three ordered parts in the result:
      \begin{enumerate}
        \item $r_1$ consumes and $r_2$ is skipped: $ms'$ is appended with $\mkeps(r_2)$
        to account for the case where $r_1$ consumes part of the string while $r_2$ is skipped.
        \item $r_1$ and $r_2$ both consume: each mark in $ms'$ is shifted into $r_2$,
        accounting for both parts consuming characters.
        (The reason why each mark is shifted individually is explained in the final case below.)
        \item $r_1$ is skipped and $r_2$ consumes: the original marks $ms$ are appended with
        $\mkeps(r_1)$ and shifted into $r_2$, accounting for the case where $r_1$ is entirely skipped.
      \end{enumerate}

    \item $\nullable(r_1)$: if $r_1$ can be skipped, each mark in $ms'$ is first shifted into $r_2$,
    and then the original list of marks $ms$, appended with $\mkeps(r_1)$, is also shifted into $r_2$
    to account for skipping $r_1$.  
    This preserves the remaining-suffix order by prioritising marks where $r_1$ consumes input
    over those where it is skipped, similar to case~(3) above.

    \item $\nullable(r_2)$: if $r_2$ can be skipped, $ms'$ is first appended with $\mkeps(r_2)$
    (accounting for skipping $r_2$ entirely), and then each mark in $ms'$ is shifted into $r_2$.  
    This preserves the remaining-suffix order by prioritising marks where $r_1$ consumes while
    $r_2$ is skipped over those where both consume.

    \item $\neg\nullable(r_1) \land \neg\nullable(r_2)$:
    if neither $r_1$ nor $r_2$ can be skipped, we shift each mark in $ms'$ individually into $r_2$.
    If we moved all marks simultaneously to the second part, we would lose the remaining-suffix order
    in cases where a list of marks from the first part does not match all subexpressions
    in the second part, which would disrupt the correct ordering.

    Consider the example $((a + b) + ab) \cdot (bc + (c + b))$ with the string \texttt{abc}:
    \begin{center}
      \begin{enumerate}
        \item $\bigl|_{[\Marked{abc}]}\;((a+b)+ab) \cdot (bc+(c+b))$
        \item $\;((a+b)+ab)\;\bigl|_{[\Marked{c},\,\Marked{bc}]}\;\cdot\;(bc+(c+b))$
      \end{enumerate}
    \end{center}

    Here, $\Marked{c}$ has consumed the leftmost match via $ab$, while $\Marked{bc}$ consumed only $a$.
    If both marks were moved simultaneously, $\Marked{c}$ would be dropped by the left subexpression 
    since it does not match $bc$, and $\Marked{bc}$—which matches successfully—would produce 
    $\Marked{\emptylist}$ and take priority in the result of the alternative $r_2$,
    since the left and right mark lists are concatenated in order.
    To avoid this issue, we shift each mark individually into $r_2$:
    first, $\Marked{c}$ is shifted, producing an empty list for the left subexpression
    and $\Marked{\emptylist}$ for the right;
    then $\Marked{bc}$ is shifted, producing $\Marked{c}$ for the left
    and $\Marked{\emptylist}$ for the right.
    Combining both results keeps the remaining-suffix order and correctly prioritises
    the POSIX value.
  \end{enumerate}
\end{itemize}
\paragraph*{\textbf{Star case} $(r^{*})$:}
\begin{itemize}
  \item Similar to the previous version, but with annotating $0$ to indicate the beginning of 
  a star iteration, and applying $\reshuffle$ to reorder the resulting list and preserve 
  the remaining-suffix order. The end of an iteration is indicated by appending $1$ 
  when returning the marks representing that iteration.
\end{itemize}

\paragraph*{\textbf{n-Repetitions case} $(r^{n})$:}
The shifting behaviour depends on several cases depending on $n$.
\begin{itemize}
  \item If $n = 0$, then no marks are shifted and an empty list is returned, following the choice
  made in the previous versions for the empty string expression $\ONE$ (see Section~\ref{sec:scala-fischer}),
  since $n = 0$ represents the same language as $\ONE$.
  \item If $n = 1$, the marks are shifted only into the inner $r$, representing one iteration.
  The $0$ is appended to indicate the beginning of the iteration, after which $\reshuffle$
  is applied to preserve the remaining-suffix order.
  \item If $n > 1$, the marks are first shifted into the inner $r$ with $0$ appended to indicate
  the beginning of the iteration, and $\reshuffle$ is applied to preserve the remaining-suffix order.
  If the result is empty, an empty list is returned. Otherwise, we have two cases:
    \begin{itemize}
      \item If $r$ is nullable, we append $1$ to indicate the end of the iteration without completing
      all $n$ repetitions due to skipping $r$, and shift the result of shifting into $r$ into $r^{n-1}$
      to represent the remaining iterations. This accounts for the possibility of skipping $r$,
      similar to the behaviour in the \Star\ case.
      \item If $r$ is not nullable, we simply shift the result into $r^{n-1}$ without appending $1$.
    \end{itemize}
\end{itemize}

%\subsection{String-Carrying Marks - Matcher - Proofs **so far**}


\end{document}