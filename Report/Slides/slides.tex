\documentclass[aspectratio=169,11pt]{beamer}
%how about we work on section and titles names. suggest a good way to organize the slides
% ====== Packages ======
\usepackage{amsmath,amssymb,amsthm}
\usepackage{booktabs,array,bm,graphicx}
\usepackage{tikz}
\usepackage{forest}
%define =
\newcommand{\dn}{\ensuremath{\stackrel{\mbox{\scriptsize def}}{=}}}
%regex
\newcommand{\ZERO}{\small\textbf{0}}   
\newcommand{\ONE}{\small\textbf{1}}    

%function definitions
\newcommand{\der}{\textit{der}}
\newcommand{\shift}{\textit{shift}}
\newcommand{\shifts}{\textit{shifts}}
\newcommand{\fuse}{\textit{fuse}}
\newcommand{\mkeps}{\textit{mkeps}}
\newcommand{\intern}{\textit{intern}}
\newcommand{\mkfin}{\textit{mkfin}}
\newcommand{\fin}{\textit{fin}}
\newcommand{\nullable}{\textit{nullable}}
\newcommand{\matcher}{\textit{matcher}}
\newcommand{\reshuffle}{\textit{reshuffle}}
\newcommand{\follow}{\textit{follow}}
\newcommand{\readF}{\textit{read}}
\newcommand{\move}{\textit{move}}
%value definitions
\newcommand{\Seq}{\textit{Seq}}
\newcommand{\Left}{\textit{Left}}
\newcommand{\Right}{\textit{Right}}
\newcommand{\Star}{\textit{Star}}
\newcommand{\Empty}{\textit{Empty}}

%symbols
\newcommand{\emptylist}{[\,]}
\newcommand{\At}{\text{$\,@\,$}}
\newcommand{\Marked}[1]{\bullet\,#1}
\newcommand{\Bits}{\textit{Bits}}
\newcommand{\POINT}{\textit{POINT}}
\newcommand{\bigO}[1]{\ensuremath{\mathop{}\mathopen{}\mathcal{O}\mathopen{}\left(#1\right)}}
\newcommand{\sizeExplosion}{"size explosion"}


% ====== Title ======
\title{9-Month Progress Report}
\author{Meshal Binnasban}
\institute{King’s College London — Department of Informatics}
\date{\today}


\begin{document}

% ------------------ Title ------------------
\begin{frame}
  \titlepage
  \note{Introduce yourself, supervisor(s), and the one-line goal: POSIX-correct value extraction via a marked approach.}
\end{frame}

 \begin{frame}
    \frametitle{Overview}
    \tableofcontents
\end{frame}

\section{Introduction}
\subsection{Brzozowski’s Derivatives}
% ------------------ Intro: Derivatives (Brzozowski) ------------------
\begin{frame}{Brzozowski’s Derivatives}
\small
\begin{itemize}
  \item The notion of derivatives for regular expression matching was introduced by 
  Brzozowski (1964).
  \item Regained interest in the last decade.\footnote{See Owens~\cite{Owens2009} 
  and Might~\cite{Might2011}.}
  \item Compute derivatives successively with respect to each input character.
  \item Acceptance for a word $w = a_1\cdots a_n$ is tested by: 
  $\varepsilon \in L(\der_{a_n}(\cdots (\der_{a_1}(r))))$.
  \item Elegant in design; can be easily implemented in functional programming 
  languages and reasoned about in theorem provers.
  \item \textbf{Challenge — Size Explosion:} successive derivatives can rapidly 
  increase the expression size, especially in sequences and Kleene stars.

\end{itemize}

\note{Emphasise the sequence rule as a source of growth.}
\end{frame}

% ------------------ Derivative rules ------------------
\begin{frame}{Brzozowski’s Derivative}
\small
%explosion example right after.
\centering
\renewcommand{\arraystretch}{1.4}
\begin{tabular}{lcl}
  $\der_a(\ZERO)$ & $\Rightarrow$ & $\ZERO$ \\[2pt]
  $\der_a(\ONE)$  & $\Rightarrow$ & $\ZERO$ \\[2pt]
  $\der_a(c)$ & $\Rightarrow$ &
  $\begin{cases}
    \ONE  & \text{if } a = c,\\
    \ZERO & \text{otherwise.}
  \end{cases}$ \\[6pt]
  $\der_a(r_1 + r_2)$ & $\Rightarrow$ & $\der_a(r_1) + \der_a(r_2)$ \\[4pt]
  $\der_a(r_1 \cdot r_2)$ & $\Rightarrow$ &
  $\der_a(r_1) \cdot r_2 +
    \begin{cases}
      \der_a(r_2) & \text{if } \nullable(r_1),\\
      \ZERO       & \text{otherwise.}
    \end{cases}$ \\[6pt]
  $\der_a(r^\ast)$ & $\Rightarrow$ & $\der_a(r) \cdot r^\ast$\\
\end{tabular}

%\vspace{0.4em}\hrule\vspace{0.2em}
\mbox{}
\rule{\linewidth}{0.4pt}

\textit{Brzozowski’s derivatives}
\note{Point at the sequence rule; the extra summand is the source of growth when $r_1$ is nullable.}
\end{frame}

% ------------------ Derivatives Matching Example ------------------
\begin{frame}{Matching Example}
\small
\textbf{Regular expression } $(ab + ba)$ \medskip \textbf{String } \textit{ba}
\[
\begin{array}{rcl}
\der_b\, r
  &=& \der_b\, (ab + ba) \\
  &=& \der_b\, (ab) + \der_b\, (ba) \\
  &=& (\der_b\, a) \cdot b + (\der_b\, b) \cdot a \\
  &=& \ZERO \cdot b + \ONE \cdot a \\[6pt]
\der_a\, (\der_b\, r)
  &=& \der_a\, (\ZERO \cdot b + \ONE \cdot a)\\
  &=& \der_a\, (\ZERO \cdot b) + \der_a\, (\ONE \cdot a)\\
  &=& \der_a\, (\ZERO) \cdot b + (\der_a\, (\ONE) \cdot a + \der_a\, a)\\
  &=& \ZERO \cdot b + (\ZERO \cdot a + \ONE )
\end{array}
\]

\note{Walk through the derivatives and finish with the nullable test.}
\end{frame}

\subsection{Size Explosion}
% ------------------ Problem with Derivatives ------------------
\begin{frame}{Size Explosion}
\small
\begin{itemize}
  \item Expression size can increase as new subexpressions are introduced, 
      particularly in sequences and Kleene stars.
  \item This repeated expansion may lead to \textbf{size explosion}, where the 
      number of derivative expressions grows substantially with input length.
  \item Simplifications can reduce redundancy but cannot fully prevent size explosion in the worst case.
  \item Subexpressions cannot always be simplified, such as when identical terms are separated by other expressions or occur at different nesting levels.
\end{itemize}
\medskip
\footnotesize
\textbf{Regular expression } $ ((a)^* + (aa)^* + (aaa)^* + (aaaa)^* + (aaaaa)^* )^*$  \textbf{String } \textit{$ \underbrace{a \dots a}_n$}

\[
\begin{array}{rcl}
\der_a\, r
  &=& ((a)^* + (a \cdot (aa)^*) + (aa \cdot (aaa)^*) + \dots) \cdot r^* \\
\der_a(\der_a\, r)
  &=& ((a)^* + (aa)^* + (a \cdot (aaa)^*) + \dots) \cdot r^* 
      \;+\; ((a)^* + (a \cdot (aa)^*) + \dots) \cdot r^* \\
  \multicolumn{3}{c}{\vdots} \\[-0.3em]
\end{array}
\]


\note{Motivate the marks approach without too much detail.}
\end{frame}

\section{Marked Approach}
\subsection{Fischer et al.’s Marked Approach}
% ------------------ Marks Approach: Fischer et al. ------------------
\begin{frame}{Fischer et al.’s Marked Approach}
\small
\begin{itemize}
\item Moves marks through the regular expression without modifying it.
\item Marks indicate the progress of matching; acceptance occurs when
      a mark reaches a final position.
\item The structure of the regular expression remains \textbf{unchanged} during matching.
\item A promising alternative to derivative- and automaton-based approaches,
      as the regular expression structure remains fixed in size.
\item Existing works focus on \textbf{matching only}, without value 
extraction.\footnote{See Fischer et~al.~\cite{Fischer2010} 
and Asperti et~al.~\cite{Asperti2010}.}
\end{itemize}

\end{frame}

% ------------------ Marks Example (2x2 layout) ------------------
\begin{frame}{Shift Example}
\footnotesize
\forestset{
  default preamble={for tree={
    draw,
    circle,
    minimum width=1em,  
    inner sep=0.2em,      
    align=left,
    s sep=1em,          
    l sep=1.5em           
  }}
}
\textbf{Regular expression } $a \cdot (b \cdot c)$ \textbf{ String } \textit{abc}

\vspace{0.6em}

\begin{columns}[c,onlytextwidth]
  % Tree 1
  \begin{column}{0.20\textwidth}
    \centering
    \begin{forest} baseline
      [$\cdot$
        [$a$]
        [$\cdot$ [$b$] [$c$]]
      ]
    \end{forest}
  \end{column}
  % Arrow a
  \begin{column}{0.05\textwidth}
    \Large $\xrightarrow{\;a\;}$
  \end{column}

  % Tree 2
  \begin{column}{0.20\textwidth}
    \centering
    \begin{forest} baseline
      [$\cdot$
        [$\Marked{a}$]
        [$\cdot$ [$b$] [$c$]]
      ]
    \end{forest}
  \end{column}

  % Arrow b
  \begin{column}{0.05\textwidth}
    \centering
    \Large $\xrightarrow{\;b\;}$
  \end{column}

  % Tree 3
  \begin{column}{0.20\textwidth}
    \centering
    \begin{forest} baseline
      [$\cdot$
        [$a$]
        [$\cdot$ [$\Marked{b}$] [$c$]]
      ]
    \end{forest}
  \end{column}

  % Arrow c
  \begin{column}{0.05\textwidth}
    \centering
   \Large $\xrightarrow{\;c\;}$
  \end{column}

  % Tree 4
  \begin{column}{0.20\textwidth}
    \centering
    \begin{forest} baseline
      [$\cdot$
        [$a$]
        [$\cdot$ [$b$] [$\Marked{c}$]]
      ]
    \end{forest}
  \end{column}
\end{columns}

\end{frame}


\subsection{Limitations \& Our Direction}
% ------------------ Limitation + Our Direction ------------------
\begin{frame}{Limitations \& Our Direction}
\small
\begin{itemize}
  \item Our main aim is to extend the marked approach (currently acceptance only) to support \textbf{POSIX value extraction} and to formally verify its correctness.
  \item One of our goals is also to handle additional constructors, such as bounded repetition and intersection.
  \item We extended marks from single-character matching to \textbf{string-carrying marks}, where each mark carries the remaining input suffix.
  \item The basic mark representation loses valuable information, such as ordering, and cannot directly handle constructors like intersection.
  \item Extending the mark structure allows this information to be retained, giving better control over ordering and disambiguation.
  \item Future direction: continue the lexing version and establish formal verification and equivalence with the derivative-based semantics.
\end{itemize}
\note{Your work extends this to value extraction (POSIX) and richer constructors.}
\end{frame}

\section{Our Work}
\subsection{String-Carrying Marks}
% ------------------ Example: shifts with String-Carrying Marks ------------------
\begin{frame}{String-Carrying Marks}
\small
\begin{itemize}
  \item Each mark carries the \textbf{remaining input suffix}.
  \item Matching begins with an initial mark containing the full input string.
  \item At each shift, a matching character is stripped from the mark’s string.
  \item A match occurs when an \textbf{empty mark} is produced, indicating full input consumption.
\end{itemize}

\note{Explain that string-carrying marks preserve input order and enable extraction of POSIX-consistent values.}
\end{frame}


\subsection{\shifts\ Definition}
% ------------------ Definition of shifts ------------------
\begin{frame}{\shifts\ Definition}
\small
\centering
\renewcommand{\arraystretch}{1.6}
\begin{tabular}{ccl}
$\shifts(ms, 0)$ & $\Rightarrow$ & $\emptylist$ \\
$\shifts(ms, 1)$ & $\Rightarrow$ & $\emptylist$ \\
$\shifts(ms, d)$ & $\Rightarrow$ & $[\, \Marked{s} \mid \Marked{d::s} \in ms \,]$ \\
$\shifts(ms, r_1 + r_2)$ & $\Rightarrow$ & $\shifts(ms, r_1) \; @ \; \shifts(ms, r_2)$ \\[2pt]
$\shifts(ms, r_1 \cdot r_2)$ & $\Rightarrow$ &
$\text{let } ms' = \shifts(ms, r_1) \text{ in}$ \\[2pt]
\multicolumn{3}{@{\hspace{10mm}}l}{
$\begin{cases}
\shifts(ms' @ ms, r_2) @ ms' & \text{if } \nullable(r_1) \land \nullable(r_2),\\
\shifts(ms' @ ms, r_2)       & \text{if } \nullable(r_1),\\
\shifts(ms', r_2) @ ms'      & \text{if } \nullable(r_2),\\
\shifts(ms', r_2)            & \text{otherwise.}
\end{cases}$}\\[2pt]
$\shifts(ms, r^{*})$ & $\Rightarrow$ & $\text{let } ms' = \shifts(ms, r) \text{ in}$\\
&&$\text{if } ms' = \emptylist \text{ then } \emptylist \text{ else } \shifts(ms', r^{*}) @ ms'$
\end{tabular}

\medskip
\textit{Each mark carries a remaining input suffix; $ms$ is a list of marks.}
\end{frame}

% ------------------ Worked example of \shifts ------------------
\begin{frame}{Matching Example}

\textbf{Regular expression } $(a + (a \cdot c))$ \medskip \textbf{String } \textit{ac}

\begin{enumerate}
  \item $\bigl|_{[\Marked{ac}]} \;( a + (a \cdot c) )$
  \item $(\bigl|_{[\Marked{ac}]} a) \;+\; (\bigl|_{[\Marked{ac}]} (a \cdot c))$
  \item $\bigl(a \;\bigl|_{[\Marked{c}]}\bigr) \;+\; \bigl((a \cdot c)\;\bigl|_{[\Marked_{\emptylist}]}\bigr)$
  \item $ [{}_{ {}_\Marked{c},{}_\Marked{\emptylist} }]$
\end{enumerate}

\note{Step through how the input string is gradually consumed and marks become empty.}
\end{frame}


\bibliographystyle{abbrv}
\bibliography{urules}

\end{document}
